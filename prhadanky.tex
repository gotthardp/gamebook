% Generated 24.8.2000 18:29:00
% Codepage ISO-8859-2

\section{Lidové hádanky}
\begin{multicols}{\value{columnsthindata}}

\noindent
Trochu voda, trochu zem,\\
volá na tě: nechoď sem!\\[1 mm]
{\sl Bláto}\\

\noindent
Které zvíře pije nejdražší nápoj na světě.\\[1 mm]
{\sl Blecha, klíště, komár}\\

\noindent
Malé, černé,\\
velkou kládou pohne.\\[1 mm]
{\sl Blecha}\\

\noindent
Jak chodí husy do vody?\\[1 mm]
{\sl Bosy}\\

\noindent
Můj počátek zná, kdo se štítí,\\
můj celek však spíš přijde k~chuti\\
mé konce, hřiby, lovce tají,\\
,,ham`` ze dvou třetin přičítají.\\[1 mm]
{\sl Br-am-bory}\\

\noindent
Je ve tmě důleček,\\
v tom důlečku kulka,\\
na té kulce chrásteček,\\
na chrástečku plno kuliček.\\[1 mm]
{\sl Brambor}\\

\noindent
Květ nahoře, plody dole,\\
nenajdeš ho ve stodole.\\
Prohlédni si sklep a spíž,\\
co to je, už asi víš.\\
A když je máš na talíři,\\
k obědu sníš nejvíc čtyři.\\[1 mm]
{\sl Brambory}\\

\noindent
Roztrhaná plachta\\
na poli se cachtá.\\[1 mm]
{\sl Brány (na poli)}\\

\noindent
Udělám ti, Vítku, hravě\\
ze zubaté bezzubou,\\
a z~bezzubé zubatou\\
udělá můj bratr v~trávě.\\[1 mm]
{\sl Brousek na kosu}\\

\noindent
Dvě hlavičky bez úst spolu hovoří,\\
na kůži když se oboří.\\[1 mm]
{\sl Buben}\\

\noindent
Přišel k~nám panáček,\\
měl červený fráček,\\
když jsme ho svlékali,\\
nad ním jsme plakali.\\[1 mm]
{\sl Cibule}\\

\noindent
Sedí panna na vrátkách\\
v~devíti kabátkách,\\
je jí zima,\\
až je siná.\\[1 mm]
{\sl Cibule}\\

\noindent
Sedí panna v~okně,\\
suknička jí mokne.\\
Přijde pan kapitán,\\
hlavičku jí utne.\\[1 mm]
{\sl Cibule}\\

\noindent
Slunce v~trávě celé dny,\\
natahuje hodiny,\\
crr - crr - crr -\\
v~noci zase měsíček\\
natahuje budíček,\\
crr - crr - crr.\\[1 mm]
{\sl Cvrček}\\

\noindent
Čím je tomu více let,\\
tím je to novější,\\
čím je tomu méně let,\\
tím je to starší.\\[1 mm]
{\sl Čas}\\

\noindent
Nevidíš to, necítíš to,\\
nemá to křídel, a přece to letí.\\[1 mm]
{\sl Čas}\\

\noindent
Šlehal sedlák do valachů,\\
roztrhl sobě pytel hrachu,\\
ztratil kozí syreček.\\[1 mm]
{\sl Červánky, hvězdy, měsíc}\\

\noindent
Roste, roste stromeček,\\
pichlavý jako bodláček,\\
kulatý jak jablíčko,\\
červený jak líčko.\\[1 mm]
{\sl Červená ředkev}\\

\noindent
Chlupaté chlupatiště uhání,\\
v~hrníčku květu má snídani.\\[1 mm]
{\sl Čmelák, včela}\\

\noindent
Letí pán přes boří,\\
sám sobě hovoří;\\
má krátké nožičky,\\
černé nohavičky.\\[1 mm]
{\sl Čmelák}\\

\noindent
Čtvero koní pod jabloní,\\
čtvero koní se tu honí.\\
Jeden je černý, druhý zelený,\\
třetí zlatý, čtvrtý bílý.\\
Povězte mi, lidé milí,\\
co je to?\\[1 mm]
{\sl Čtvero ročních dob}\\

\noindent
Mám otce i matku\\
a přece nejsem syn.\\[1 mm]
{\sl Dcera}\\

\noindent
Běhá to okolo chalupy\\
a dělá cupity dupity.\\[1 mm]
{\sl Déšť}\\

\noindent
Co neshnije na dřevěné střeše.\\[1 mm]
{\sl Díra}\\

\noindent
Kdo mne má v~kapse,\\
nemá v~ní nic.\\[1 mm]
{\sl Díra}\\

\noindent
Když k~tomu přidáš, je to menší,\\
když od toho odebereš, je to větší.\\[1 mm]
{\sl Díra}\\

\noindent
Čím je jich více,\\
tím méně váží.\\[1 mm]
{\sl Díry}\\

\noindent
Masná hůra,\\
dřevěná důla:\\
shůry do důly\\
čtyři stříkají.\\[1 mm]
{\sl Dojení}\\

\noindent
Do domečku oknem vleze,\\
pak se s~ostatními sveze.\\[1 mm]
{\sl Dopis, poštovní schránka}\\

\noindent
Běží posel beznohý,\\
jest on šelma čtverrohý.\\
Nikomu nic nepoví,\\
přece všechno vypoví.\\[1 mm]
{\sl Dopis}\\

\noindent
Co dělá husa, když o jedné noze stojí?\\[1 mm]
{\sl Druhou zdvíhá}\\

\noindent
Otec má na tisíce synů,\\
každému čepici zjedná\\
a sobě nemůže.\\[1 mm]
{\sl Dub a žaludy}\\

\noindent
Dlouhá je a trvá krátkou chvíli,\\
široká je víc než jednu míli,\\
vysoká je, má sedmery schody,\\
přec se vejde do kapičky vody.\\[1 mm]
{\sl Duha}\\

\noindent
Když se otec směje,\\
matka slzy leje.\\
Dcerka ze sedmeré krásy\\
jasné pásy, obléká si.\\[1 mm]
{\sl Duha}\\

\noindent
Sto luk, na každé louce sto kop, u každé kopy sto volů, kolik
je to rohů ?\\[1 mm]
{\sl Dva milióny}\\

\noindent
Čtyři rohy, žádné nohy.\\
Chodí to, stojí to.\\[1 mm]
{\sl Dveře}\\

\noindent
Vrzy sem, vrzy tam,\\
z~místa přec nikam.\\[1 mm]
{\sl Dveře}\\

\noindent
Celá chalupa vyhoří,\\
a přece se nezboří.\\[1 mm]
{\sl Dýmka}\\

\noindent
Cpou do ní jak do stodoly,\\
dechem svým vyhání moly.\\[1 mm]
{\sl Dýmka}\\

\noindent
Mihla se, už tu není,\\
zvědavá mašlička na kameni.\\[1 mm]
{\sl Had}\\

\noindent
Odťata od krku,\\
a přece krev neteče.\\[1 mm]
{\sl Hlávka zelí}\\

\noindent
Čtyři rohy,\\
žádné nohy,\\
chaloupkou to pohne.\\[1 mm]
{\sl Hlemýžď}\\

\noindent
Chodí každý den ven,\\
a přece je stále doma.\\[1 mm]
{\sl Hlemýžď}\\

\noindent
Neutíká, nezakluše,\\
pomalu se loudá.\\
Na zádech mu místo nůše\\
sedí celá bouda.\\[1 mm]
{\sl Hlemýžď}\\

\noindent
Znám já chalupníka,\\
co chaloupku svou nikdy neprodá.\\[1 mm]
{\sl Hlemýžď}\\

\noindent
Dvanáct panen v~jedné posteli,\\
a žádná na okraji.\\[1 mm]
{\sl Hodiny}\\

\noindent
Jde panáček cestou,\\
chvíli před nevěstou, chvíli za nevěstou.\\
Z~dálky na ně kmotr kývá,\\
jim však cesty neubývá.\\[1 mm]
{\sl Hodiny}\\

\noindent
Jde stařeček cestou,\\
a nic mu cesty neubývá.\\[1 mm]
{\sl Hodiny}\\

\noindent
Každodenně se natahují,\\
a přece jsou stále stejně dlouhé.\\[1 mm]
{\sl Hodiny}\\

\noindent
Malý domeček\\
plný koleček.\\[1 mm]
{\sl Hodiny}\\

\noindent
Nejí to, nepije to,\\
ve dne v~noci chodí to.\\[1 mm]
{\sl Hodiny}\\

\noindent
Visí to a neví kde,\\
bije to a neví koho,\\
ukazuje to a neví kam,\\
počítá a neví kolik.\\[1 mm]
{\sl Hodiny}\\

\noindent
Dřevěnou má v~břiše duši,\\
dlouhý krk a čtyři uši.\\[1 mm]
{\sl Housle}\\

\noindent
Táta dlouhý, máma krátká,\\
děti jako pimprlátka.\\[1 mm]
{\sl Hrábě, ruka}\\

\noindent
Ze země pošel, krmil, napájel,\\
a když umřel, ani ho nepochovali.\\[1 mm]
{\sl Hrnec (hliněný)}\\

\noindent
Když šel tam do hlavy ho tloukli,\\
když šel ven, za krk ho tahali.\\[1 mm]
{\sl Hřebík}\\

\noindent
Má to klobouček,\\
jednu nožičku,\\
pěkně si sedí\\
v~mechu v~lesíčku.\\[1 mm]
{\sl Hřib, houba}\\

\noindent
Má pěknou hlavičku,\\
jen jednu nožičku,\\
hoví si v~lesíčku.\\[1 mm]
{\sl Hřib}\\

\noindent
Sedí pán v~dubí,\\
kloboukem se chlubí.\\[1 mm]
{\sl Hřib}\\

\noindent
Široký já deštník mám,\\
po dešti jej rozvírám.\\[1 mm]
{\sl Hřib}\\

\noindent
Stojí, stojí žebrák,\\
má záplatu na záplatě,\\
ani šos mu neznat.\\[1 mm]
{\sl Husa}\\

\noindent
Na poli žaté,\\
v~lese ťaté,\\
u kováře dělané,\\
u sedláře také.\\[1 mm]
{\sl Chomout}\\

\noindent
Když mně přidávají, menší jsem;\\
když mně ubírají, větší jsem.\\[1 mm]
{\sl Jáma}\\

\noindent
Červený beránek po ovčíně skáče.\\[1 mm]
{\sl Jazyk, zuby}\\

\noindent
Kdo vyleze na patro bez žebříku?\\[1 mm]
{\sl Jazyk}\\

\noindent
Stojí, stojí, kmotr zlatý,\\
vousatý a kolenatý.\\[1 mm]
{\sl Ječmen (zralý)}\\

\noindent
Je švadlena u vesnice,\\
co má jehel na tisíce.\\[1 mm]
{\sl Jedle}\\

\noindent
Ouško mám a neslyším,\\
nožku mám a neběhám.\\[1 mm]
{\sl Jehla}\\

\noindent
Železné ptáče\\
přes ploty skáče\\
a koudelný ocas má.\\[1 mm]
{\sl Jehla}\\

\noindent
Běží krejčík po poli,\\
a na hřbetě si nese jehly.\\[1 mm]
{\sl Ježek}\\

\noindent
Cupe, dupe v~lese,\\
nůši jehel nese.\\[1 mm]
{\sl Ježek}\\

\noindent
Jede, jede lovec,\\
má červený konec.\\
Kam ten konec strčí,\\
všude voda crčí.\\[1 mm]
{\sl Kačer}\\

\noindent
Jede, jede panáček,\\
má žlutý zobáček,\\
kde vodička hrčí,\\
tam zobáček strčí.\\[1 mm]
{\sl Kačer}\\

\noindent
Denně to používáte,\\
chcete po tom vždy to samé,\\
odpověď je však pokaždé jiná.\\[1 mm]
{\sl Kalendář}\\

\noindent
Jednou dírou tam, dvěmi dírami ven,\\
a když jste venku, teprve jste tam.\\[1 mm]
{\sl Kalhoty}\\

\noindent
Běží ovce, běží,\\
u nohou jim sněží,\\
a pod vodou jehňata\\
svítí jako ze zlata.\\[1 mm]
{\sl Kameny a skály v~řece}\\

\noindent
Stojí, stojí sloup,\\
do sloupu se dá hořalka,\\
na hořalku dřevěnka,\\
z~dřevěnky teploučko.\\[1 mm]
{\sl Kamna}\\

\noindent
Jak dlouho nosí kůň podkovu\\[1 mm]
{\sl Když má nohu nahoře (jinak nese podkova koně)}\\

\noindent
Bílé jsou a černé schůdky,\\
prsty mají s~nimi půtky.\\[1 mm]
{\sl Klavír, klávesy}\\

\noindent
Máme doma shrbeného stařečka,\\
podáváme mu často ruku.\\[1 mm]
{\sl Klika}\\

\noindent
Od roka do roka\\
máme my proroka,\\
jísti mu nedáváme,\\
ruku mu podáváme.\\[1 mm]
{\sl Klika}\\

\noindent
Sedí pán na vršku,\\
měkkou má podušku,\\
často se smekne,\\
nic se nelekne.\\[1 mm]
{\sl Klobouk}\\

\noindent
Je velké jako jablíčko,\\
a kůň to z~jámy nevytáhne.\\[1 mm]
{\sl Klubíčko}\\

\noindent
Listy mám a strom nejsem.\\[1 mm]
{\sl Kniha}\\

\noindent
Přišel k~nám voják, měl červený kabát,\\
bílé tělo a černou duši.\\[1 mm]
{\sl Kobliha}\\

\noindent
Má to hlavu jako kočka,\\
má to nohy jako kočka,\\
má to ocas jako kočka,\\
mňouká to jako kočka,\\
a není to kočka.\\[1 mm]
{\sl Kocour}\\

\noindent
Čtyři stelou, dva svítí,\\
jeden si lehne.\\[1 mm]
{\sl Kocouří nohy, oči a ocas}\\

\noindent
Tázala se hryzka,\\
je-li chňapka doma ?\\[1 mm]
{\sl Kočka a myš}\\

\noindent
Na krbu sedí,\\
však nevaří,\\
přede,\\
ale přádlu se nedaří.\\[1 mm]
{\sl Kočka}\\

\noindent
Chodí v~koruně, král není,\\
nosí ostruhy, rytíř není,\\
má šavli, husar není,\\
k~ránu budívá, ponocný není.\\[1 mm]
{\sl Kohout}\\

\noindent
Běží čtyři bratři bez nohou,\\
jeden druhého dohnat nemohou.\\[1 mm]
{\sl Kola u vozu}\\

\noindent
Čtyři pacholíci se honí,\\
žádný žádného nedohoní.\\[1 mm]
{\sl Kola u vozu}\\

\noindent
Dřevo bylo, listy mělo,\\
nese duši, nese tělo.\\[1 mm]
{\sl Kolébka}\\

\noindent
Kolik bývá na stromě listů ?\\[1 mm]
{\sl Kolik stopek}\\

\noindent
Ani zvíře, ani pták,\\
místo nosu špikovák.\\
Letí, ječí,\\
sedne, mlčí.\\
Kdo ho zabije,\\
svou krev prolije.\\[1 mm]
{\sl Komár}\\

\noindent
Máme staříčka,\\
co kouřívá třikrát za den.\\[1 mm]
{\sl Komín}\\

\noindent
Na peci sedí,\\
do nebe hledí.\\
Někdy si zahučí,\\
zavyje a zaskučí,\\
kočky to vědí.\\[1 mm]
{\sl Komín}\\

\noindent
Stojím na peci, a přece moknu na dešti,\\
mám v~břichu oheň, a přece mrznu.\\[1 mm]
{\sl Komín}\\

\noindent
Když se břicho s~krkem skloní,\\
hlava steré slzy roní.\\[1 mm]
{\sl Konev (kropící)}\\

\noindent
Běžela lvice\\
od Kamenice,\\
kam pohlédla,\\
tráva zbledla.\\[1 mm]
{\sl Kosa}\\

\noindent
Narodila jsem se na stromě,\\
máti mě dala lidem,\\
lidi mne z~kůže stáhli,\\
potom mne svázali,\\
po zemi tahali,\\
až mě odřeli a do ohně hodili.\\[1 mm]
{\sl Koště, metla}\\

\noindent
Pannou v~lese,\\
babou v~domě;\\
hlavou dole,\\
nohy nahoře;\\
po světnici běhá,\\
pod lavicí spává.\\[1 mm]
{\sl Koště, pometlo}\\

\noindent
Běhá po světnici,\\
lehá pod lavicí.\\[1 mm]
{\sl Koště}\\

\noindent
Přišel k~nám host,\\
co v~lese vzrost,\\
zatočil se po světnici,\\
praštil sebou pod lavici.\\[1 mm]
{\sl Koště}\\

\noindent
Nikde konce ani počátku,\\
běhá dopředu i pozpátku.\\[1 mm]
{\sl Koule}\\

\noindent
Jde červený žváč,\\
má dva roháče,\\
čtyři cupitáče,\\
sedmý ometáč.\\[1 mm]
{\sl Kráva}\\

\noindent
Co dělá hodný žák, když jde do školy?\\[1 mm]
{\sl Kroky}\\

\noindent
Zpívám,\\
ale nemám úst,\\
sám a sám\\
se bojím růst.\\[1 mm]
{\sl Les}\\

\noindent
Beznohý na dřevo leze,\\
bezruký na něj hází,\\
němý po cestě si brouká,\\
hluchý v~trní ho poslouchá.\\
Co je to?\\[1 mm]
{\sl Lež}\\

\noindent
Maličké, slaďoučké,\\
v~okovaném hrníčku.\\[1 mm]
{\sl Lískový ořech}\\

\noindent
Je chlíveček bez dvířek,\\
v~něm je víc než tisíc oveček.\\[1 mm]
{\sl Makovice, mák}\\

\noindent
Stojí v~poli hůlka,\\
na té hůlce kulka\\
a v~té kulce plno kuliček.\\[1 mm]
{\sl Makovice}\\

\noindent
Pán s~paní hovoří,\\
div komoru nezboří.\\[1 mm]
{\sl Máselnice}\\

\noindent
Kdo nosí kožich i v létě?\\[1 mm]
{\sl Medvěd}\\

\noindent
Šlehal sedlák do valachů,\\
roztrhl sobě pytel hrachu,\\
ztratil kozí syreček.\\[1 mm]
{\sl Měsíc a hvězdy}\\

\noindent
Pole neměřené,\\
ovce nesčítané,\\
pastýř nesjednaný.\\[1 mm]
{\sl Měsíc, hvězdy, nebe}\\

\noindent
Co se leskne v~rybníce,\\
komu to spadla čepice ?\\
Třpytí se, leskne u sítí,\\
nikdo ji rukou nechytí.\\[1 mm]
{\sl Měsíc}\\

\noindent
Obchází se srpem celou zem,\\
nesekne do trávy tam ani sem.\\[1 mm]
{\sl Měsíc}\\

\noindent
Přiletěl obrázek, poseděl,\\
složil se, rozložil a odletěl.\\[1 mm]
{\sl Motýl}\\

\noindent
Šije mokrou nití,\\
k~zemi nebe stahuje,\\
když dojdou nitě,\\
tak se odstěhuje.\\[1 mm]
{\sl Mrak}\\

\noindent
Přišel k~nám host,\\
spravil nám most,\\
bez sekery bez dláta,\\
a přece je pevný dost.\\[1 mm]
{\sl Mráz}\\

\noindent
Sedí panenka v~komoře,\\
má vlasy venku na dvoře.\\[1 mm]
{\sl Mrkev, řepa}\\

\noindent
Za kadeřavou hlavičku\\
vytáhnu z nory lištičku.\\
Sáhni si, je hladká,\\
ukousni, je sladká.\\[1 mm]
{\sl Mrkev}\\

\noindent
Zelená jsem, tráva nejsem;\\
žlutá jsem, vosk nejsem;\\
ocas mám, pes nejsem.\\[1 mm]
{\sl Mrkev}\\

\noindent
Ve kterém moři není ani kapka vody?\\[1 mm]
{\sl Na mapě}\\

\noindent
Kde leží králík nejjistěji.\\[1 mm]
{\sl Na pekáči}\\

\noindent
Kde má slepice nejméně peří\\[1 mm]
{\sl Na pekáči}\\

\noindent
Maličký domeček\\
a v~něm sto okéneček.\\[1 mm]
{\sl Náprstek}\\

\noindent
Přišel pán z~Železnice,\\
poďubaly ho neštovice.\\[1 mm]
{\sl Náprstek}\\

\noindent
Máme volka,\\
celý vejde do chléva,\\
jen rohy nechá venku.\\[1 mm]
{\sl Nebozez}\\

\noindent
Dračí křídla, tělo myší,\\
špatně vidí, dobře slyší.\\
Ve dne spí a v noci létá,\\
do vlasů prý rád se vplétá.\\[1 mm]
{\sl Netopýr}\\

\noindent
Mrtví mě jedí každý den.\\
Živí, když mě pozřou, pomalu zemřou.\\[1 mm]
{\sl Nic (žádné jídlo)}\\

\noindent
Železné zvíře\\
otavu hryže\\
na živé hoře.\\[1 mm]
{\sl Nůžky, vlasy, hlava}\\

\noindent
Dvě kavky vedle sebe sedí,\\
jedna druhou nevidí.\\[1 mm]
{\sl Oči}\\

\noindent
Kdo ho spatří,\\
nechť ho potlačí;\\
kdo ho chce chovati,\\
musí se varovati;\\
dáš-li mu masti,\\
bude růsti,\\
když naroste,\\
bude krásti,\\
všecko ti před očima pobere\\
a z~domu tě vyžene.\\[1 mm]
{\sl Oheň}\\

\noindent
Když jsem v~kamnech,\\
rád mě míváš,\\
když jsem na střeše,\\
smutně se díváš.\\[1 mm]
{\sl Oheň}\\

\noindent
Měkké dělám z~tvrdého,\\
tvrdé dělám z~měkkého;\\
s~větry se rád pojím,\\
vody se však bojím.\\[1 mm]
{\sl Oheň}\\

\noindent
Rád jím dříví,\\
celý les mě živí,\\
moje zrní\\
ujme se i v~trní.\\[1 mm]
{\sl Oheň}\\

\noindent
Co se v chladnu nejvíc zapotí?\\[1 mm]
{\sl Okno}\\

\noindent
Čtyři panny v~kolébce,\\
žádná na okraji.\\[1 mm]
{\sl Ořech}\\

\noindent
Češe se, vlasů nemá,\\
klátí se, třtina není,\\
třese se, zimnici nemá.\\[1 mm]
{\sl Ovoce}\\

\noindent
Ráno obírej mne,\\
večír očesej mne,\\
jen mne neuraz,\\
dám ti za rok zas!\\[1 mm]
{\sl Ovocný strom}\\

\noindent
Skáče, skáče,\\
do stěny hlavou bije,\\
ale nezapláče.\\[1 mm]
{\sl Palička od bubnu}\\

\noindent
Stojí paní na palouku,\\
bílý závoj na klobouku.\\[1 mm]
{\sl Pampeliška}\\

\noindent
Bílé pole, černé símě,\\
kdo mě sije, rozumí mně.\\[1 mm]
{\sl Papír, písmo}\\

\noindent
Kdo má nejvíc očí na světě?\\[1 mm]
{\sl Páv}\\

\noindent
Koruna na hlavě,\\
andělský šat,\\
zlodějský krok,\\
ďábelský křik.\\[1 mm]
{\sl Páv}\\

\noindent
Tenhle pán či šlechtic snad,\\
denně rozkládá svůj šat.\\
Rozloží ho, pootočí,\\
nespočítáš jeho oči.\\[1 mm]
{\sl Páv}\\

\noindent
Spadla plachta shůry,\\
roztrhla se v~půli,\\
žádná švadlena ji nezašije.\\[1 mm]
{\sl Pavučina}\\

\noindent
Máme chlíveček,\\
je plný dřevěných slepiček,\\
když mezi ně černý kohout vletí,\\
všecky je vyžene.\\[1 mm]
{\sl Pec, uhlíky, pohrabáč}\\

\noindent
Maličký, však pevný domeček\\
skrývá stromeček.\\[1 mm]
{\sl Pecka}\\

\noindent
Neorala jsem, nesnila jsem.\\
Co je klas, to sotva vím,\\
ze všech nejvíc chleba sním.\\[1 mm]
{\sl Pec}\\

\noindent
Běží, běží, až se třese,\\
bílý šátek v ruce nese.\\[1 mm]
{\sl Pěna na vodě}\\

\noindent
Čtyři rohy, žádné nohy.\\
Jenom břicho veliké.\\[1 mm]
{\sl Peřina}\\

\noindent
Jde se koupat,\\
a nechá břicho doma.\\[1 mm]
{\sl Peřina}\\

\noindent
V~jednom rožku trošku,\\
v~druhém rožku trošku,\\
v třetím rožku trošku,\\
v~prostředku nejvíc.\\[1 mm]
{\sl Peřina}\\

\noindent
Čichá, čichá, nemá nosu,\\
co jí přijde pod zuby,\\
to semele v~otruby.\\[1 mm]
{\sl Pila}\\

\noindent
Jde panna uličkou,\\
trousí drobnou krupičkou.\\[1 mm]
{\sl Pila}\\

\noindent
Jen jedna řada zubů,\\
a přece tuze kousá.\\[1 mm]
{\sl Pila}\\

\noindent
Matka v~zemi odpočívá,\\
tatík do nebe se dívá,\\
syn po světě lidi bláznívá.\\[1 mm]
{\sl Pivo, chmel a ???}\\

\noindent
Otec se rodí,\\
syn po hůře chodí.\\[1 mm]
{\sl Plamen, dým}\\

\noindent
Kolik má pes zubů.\\[1 mm]
{\sl Plnou hubu}\\

\noindent
Po čem padá déšť ?\\[1 mm]
{\sl Po kapkách}\\

\noindent
Deset ptáčků po zlaťáčku;\\
po čem jeden?\\[1 mm]
{\sl Po noci je den}\\

\noindent
Tahle holka modrooká,\\
pořád sedá u potoka,\\
když si nohy trochu smočí\\
sluníčko jí svítí z~očí.\\[1 mm]
{\sl Pomněnka}\\

\noindent
Z~dřeva je a ze slámy\\
a přec jej oheň nestráví.\\[1 mm]
{\sl Popel}\\

\noindent
Stoupá bez nohou,\\
bublá - úst nemá,\\
bere bez rukou i kupy sena.\\[1 mm]
{\sl Potok}\\

\noindent
Škopíček bezedný,\\
plný masa nacpaný.\\[1 mm]
{\sl Prsten}\\

\noindent
Půl deváté\\
bez půl páté,\\
dvě a půl třetí,\\
kolik je to?\\[1 mm]
{\sl Půl deváté}\\

\noindent
Černý do koupele,\\
červený z~koupele.\\[1 mm]
{\sl Rak}\\

\noindent
Divné zvíře vidím v~díře,\\
na hlavě růžky a v~rukou nůžky.\\[1 mm]
{\sl Rak}\\

\noindent
Kdo to dělá, nechce to,\\
kdo to koupí, nepotřebuje to,\\
kdo to potřebuje, neví o tom.\\[1 mm]
{\sl Rakev}\\

\noindent
Stojí buk uprostřed luk,\\
na tom buku dvanáct suků,\\
v~každém suku třicet ptáků,\\
a každý pták jinak zpívá.\\[1 mm]
{\sl Rok, měsíce, dny}\\

\noindent
Brzy ráno drahokam. Kde je?\\
Byl tam, není tam.\\[1 mm]
{\sl Rosa}\\

\noindent
Denice to roztrousila,\\
slunko sebralo,\\
měsíc vrátil.\\[1 mm]
{\sl Rosa}\\

\noindent
Padám z~nebe, svítím,\\
objímám se s~kvítím,\\
nejsem hvězda, létavice,\\
nejsem sama, je nás více.\\[1 mm]
{\sl Rosa}\\

\noindent
Na dvou ratolestech deset větviček drží jeden kmen.\\[1 mm]
{\sl Ruce, prsty, tělo}\\

\noindent
Čtyři bratři u stolu, \\
pátý sedí na rohu.\\[1 mm]
{\sl Ruka}\\

\noindent
Dům hučí, hospodář v~něm mlčí.\\[1 mm]
{\sl Ryba ve vodě}\\

\noindent
Jedna paní němá,\\
v~bytě stání nemá.\\
Jakmile však svůj byt ztratí,\\
životem tu ztrátu platí.\\[1 mm]
{\sl Ryba}\\

\noindent
Přišli páni bez pozvání,\\
vzali z~domu hospodáře\\
a dům vyběhl oknem ven.\\[1 mm]
{\sl Rybáři, ryba, voda, síť}\\

\noindent
Běžím, běžím,\\
nemám dech,\\
přitom ležím\\
na zádech.\\[1 mm]
{\sl Řeka}\\

\noindent
Je had, v~těle samá díra,\\
nejí, přece neumírá.\\[1 mm]
{\sl Řetěz}\\

\noindent
Když to leží, mlčí to,\\
když to zvedneš, křičí to.\\[1 mm]
{\sl Řetěz}\\

\noindent
Čtyři nožky, dva trnožky,\\
v~zimě běží, v~létě leží.\\[1 mm]
{\sl Sáně}\\

\noindent
Ve mně tma, venku světlo;\\
ve mně zima, venku teplo;\\
ve mně teplo, venku mráz;\\
uhodne to někdo z~Vás?\\[1 mm]
{\sl Sklep}\\

\noindent
Hřebínek má, nečeše se,\\
ale večeři ti snese.\\[1 mm]
{\sl Slepice}\\

\noindent
Chodí panna po městě,\\
sukniček má na dvě stě.\\[1 mm]
{\sl Slepice}\\

\noindent
Metu, metu, nevymetu,\\
nesu, nesu, nevynesu,\\
až čas přijde samo vyjde.\\[1 mm]
{\sl Sluneční světlo}\\

\noindent
Křídla mám a nejsem pták,\\
letím, letím do oblak.\\
A když v~letu zaburácím,\\
rozlétnou se všichni ptáci\\
nad loukami nad lesem.\\
Hádej, hádej, kdo že jsem.\\[1 mm]
{\sl Snad už tě to napadlo, jsem tryskové letadlo.}\\

\noindent
Sletěl pták bezperák\\
na náš strom bezlisťák,\\
přišlo na něj bezzubátko,\\
sežralo to bezpeřátko.\\[1 mm]
{\sl Sníh a slunce}\\

\noindent
Celého tě obleku\\
bíle jako do krupice,\\
a když přijdeš do světnice,\\
honem z~tebe uteku.\\[1 mm]
{\sl Sníh}\\

\noindent
Přiletěl k~nám bílý pták,\\
zalehl nám celý sad.\\[1 mm]
{\sl Sníh}\\

\noindent
Přišel k~nám bílý kůň,\\
zalehl nám celý dvůr.\\[1 mm]
{\sl Sníh}\\

\noindent
Kdo umí bít a přece nemá ruce?\\[1 mm]
{\sl Srdce}\\

\noindent
Shrbený mužík běhá po poli,\\
nepřijde domů, až je oholí.\\[1 mm]
{\sl Srp}\\

\noindent
Sedí dvojnoh na trojnohu,\\
v~ruce drží jedno nohu.\\
Přiběh z~venku čtyřnoh, ouha,\\
vzal dvojnohu jednonoha.\\
A ten dvojnoh, kdo to poví,\\
utrh nohu trojnohovi.\\
A než se dal čtyřnoh v~běh,\\
měl tu nohu na zádech.\\[1 mm]
{\sl Stehno, švec, verpánek, pes}\\

\noindent
Je to vidět, ale není to.\\[1 mm]
{\sl Stín}\\

\noindent
Nemá rukou, ani nohou,\\
a vyleze na věž.\\[1 mm]
{\sl Stín}\\

\noindent
Samo to do jámy vleze,\\
a kdyby přišlo tisíc koní,\\
nevytáhnou toho,\\
až to samo vyleze.\\[1 mm]
{\sl Stín}\\

\noindent
Špičatý, placatý,\\
lepí se ti na paty,\\
na slunci mu neutečeš,\\
hlavičku mu neučešeš.\\[1 mm]
{\sl Stín}\\

\noindent
Vedle tebe stojí, vedle leží,\\
chytneš jej však stěží.\\[1 mm]
{\sl Stín}\\

\noindent
Když bylo třeba,\\
byl jsem plný chleba,\\
a když chleba vynesli,\\
dobrý jsem vám do jeslí.\\[1 mm]
{\sl Stoh}\\

\noindent
Jede kočí, má sto očí,\\
do každého něco strčí.\\[1 mm]
{\sl Struhadlo}\\

\noindent
Lípa, na té lípě konopa, na té konopě hlína, na té hlíně
zelina a v~zelině sviňa.\\[1 mm]
{\sl Stůl, ubrus, miska, zelí, vepřové maso}\\

\noindent
Čtyři nohy, čtyři rohy,\\
hlava žádná.\\[1 mm]
{\sl Stůl}\\

\noindent
Viděli jsme na polici\\
ležet kulaťoučkou minci.\\
Blýskala se blýskala,\\
zvednout se však nedala.\\[1 mm]
{\sl Suk ve dřevě}\\

\noindent
Mladá je velká,\\
stará je malá.\\[1 mm]
{\sl Svíčka}\\

\noindent
Břicho plné krup,\\
a na hlavě strup.\\[1 mm]
{\sl Šípek}\\

\noindent
Červený hrneček,\\
plný krupeček,\\
na vrchu střapeček.\\[1 mm]
{\sl Šípek}\\

\noindent
Na poli je kabela,\\
v~té kabele plno kabéleček.\\[1 mm]
{\sl Šípek}\\

\noindent
Sedí panna v~širém poli\\
v~červené kamizole,\\
na hlavě má chocholíček,\\
břicho plné homoliček.\\[1 mm]
{\sl Šípek}\\

\noindent
Na našem dvoře leží stařeček,\\
má tolik ran, kolik v~světě vran.\\[1 mm]
{\sl Špalek na štípání}\\

\noindent
Vaří se to, peče se to,\\
nejí se to.\\[1 mm]
{\sl Špejle v~jitrnici}\\

\noindent
Kopyta mám,\\
přec nejsem zvíře.\\[1 mm]
{\sl Švec}\\

\noindent
Stojí, stojí hůrka,\\
na té hůrce kulka,\\
na té kulce lesíček,\\
pod lesíčkem mrky,\\
pod mrkama smrky,\\
pod smrkama hamy,\\
pod hamama bery,\\
pod berama kleky,\\
pod klekama chody,\\
pod chodama černá zem.\\[1 mm]
{\sl Tělo lidské}\\

\noindent
Kdo mé jméno vysloví,\\
hned mne zničí.\\[1 mm]
{\sl Ticho}\\

\noindent
Když je viděti, nevidí mne nikdo,\\
a když není viděti, vidí mne všecko.\\[1 mm]
{\sl Tma}\\

\noindent
Ve dne malá jako myš,\\
v~noci všechno přerostu,\\
když mě vidíš,\\
nevidíš.\\[1 mm]
{\sl Tma}\\

\noindent
Malušičké, černušičké,\\
na chodníčku: jejda!\\[1 mm]
{\sl Trn}\\

\noindent
Tenounká panenka, na špičce tancuje,\\
a čím déle tancuje, tím je menší.\\[1 mm]
{\sl Tužka}\\

\noindent
Červený žije, černý umírá.\\[1 mm]
{\sl Uhel}\\

\noindent
Sedí panna v~kvítí,\\
celičká se svítí;\\
dáš-li panně polena,\\
vyžene ven jelena.\\[1 mm]
{\sl Uhlíky, dřevo, plamen}\\

\noindent
Maličký sklípek,\\
v~něm dvě řady slípek\\
a červený kohoutek.\\[1 mm]
{\sl Ústa, zuby, jazyk}\\

\noindent
Znám jednu komůrku,\\
jsou v~ní dvě řady\\
bílých tovaryšů.\\
Nikdy tam neprší\\
a přece je tam pořád mokro.\\[1 mm]
{\sl Ústa}\\

\noindent
Pěkný domek obílený,\\
ani oken, ani dveří.\\[1 mm]
{\sl Vajíčko}\\

\noindent
Soudeček bez obručí,\\
a v~něm dvojí víno.\\[1 mm]
{\sl Vajíčko}\\

\noindent
Šídlo, bodidlo -\\
po světě chodilo\\
a domů nosilo.\\[1 mm]
{\sl Včela}\\

\noindent
Je soudeček bez obrouček,\\
je v~něm dvojí víno,\\
a přece se nesmíchá.\\[1 mm]
{\sl Vejce}\\

\noindent
Chodí to po dvoře,\\
hned leží, hned oře.\\[1 mm]
{\sl Vepř}\\

\noindent
Které zvíře je člověku nejvěrnější.\\[1 mm]
{\sl Veš}\\

\noindent
Nemá to huby, ale tři zuby,\\
u jídla slouží, po něm netouží.\\[1 mm]
{\sl Vidlička}\\

\noindent
Čím je to větší,\\
tím spíš se to protáhne i malinkou dírkou.\\[1 mm]
{\sl Vítr}\\

\noindent
Do šatů mě nabíráš,\\
pak přede mnou zavíráš,\\
v~teple pro mne slzí oči,\\
vše se za mnou venku točí,\\
beru z~hlavy klobouky,\\
nepouštěj mne do mouky.\\[1 mm]
{\sl Vítr}\\

\noindent
Na vodě se točí,\\
noc se neomočí.\\[1 mm]
{\sl Vítr}\\

\noindent
Pán v~červeném klobouce\\
rozhlíží se po louce,\\
slunci nikdy nesmeká,\\
ale větru, ale větru\\
smekne zdaleka\\[1 mm]
{\sl Vlčí mák}\\

\noindent
Které zvíře je vlku nejpodobnější?\\[1 mm]
{\sl Vlčice}\\

\noindent
Panenka si u vlečky\\
plete bílé kraječky,\\
stále plete,\\
neuplete,\\
stále párá,\\
nerozpárá.\\[1 mm]
{\sl Vlna}\\

\noindent
Kdo je největší lenoch na světě.\\[1 mm]
{\sl Voda (než aby šla 5 minut do kopce, radši si 2 hodinky zajde)}\\

\noindent
Pán s~paní šepce,\\
dělají se čepce.\\[1 mm]
{\sl Voda vaří}\\

\noindent
Běží to v~potoku, od roku do roku,\\
vždycky dopředu a nikdy dozadu.\\[1 mm]
{\sl Voda}\\

\noindent
Sedí víla zelená\\
a koupá si kolena,\\
má na tisíc jazýčků,\\
ale šeptá, jenom šeptá\\
pořád stejnou písničku.\\[1 mm]
{\sl Vrba}\\

\noindent
Stojí babice\\
rozcuchaná velice.\\
Když přijdou časy,\\
zelené má vlasy,\\
když tělo chřadne,\\
do vody padne.\\[1 mm]
{\sl Vrba}\\

\noindent
Čtyři dupity tance,\\
dva vopletence,\\
jeden vometáč.\\[1 mm]
{\sl Vůl}\\

\noindent
Železná boudička,\\
dvířka jen maličká,\\
často je otvírá\\
se zoubky hlavička.\\[1 mm]
{\sl Zámek u dveří}\\

\noindent
Černá hlavička, bílá nožička.\\
Hlavička když zčervená,\\
konec nožky znamená.\\[1 mm]
{\sl Zápalka}\\

\noindent
Máš-li mne, nedbáš mně,\\
ztratíš-li mne, poznáš mne.\\[1 mm]
{\sl Zdraví}\\

\noindent
Kdo je nejrychlejším malířem?\\[1 mm]
{\sl Zrcadlo}\\

\noindent
Co roste kořenem vzhůru?\\[1 mm]
{\sl Zub (horní)}\\

\noindent
Skálami chléb drobí\\
a metličkou do studánky zametá.\\[1 mm]
{\sl Zuby a jazyk}\\

\noindent
Okolo porubu,\\
hejno bílých holubů.\\[1 mm]
{\sl Zuby v~ústech}\\

\noindent
Cínové zvíře\\
v~kamenném chlívě\\
třikrát za den řehce\\
a žrát se mu nechce.\\[1 mm]
{\sl Zvon}\\

\noindent
Koruna bez hlavy,\\
bez jazyka rozpráví,\\
necítí, nežije\\
a přec mu srdce bije.\\[1 mm]
{\sl Zvon}\\

\noindent
Mezi dvěma horama\\
bije beran rohama.\\[1 mm]
{\sl Zvon}\\

\noindent
V~kamenném chlívě,\\
bučí tam týle\\
a je ho slyšet\\
až na půl míle.\\[1 mm]
{\sl Zvon}\\

\noindent
Sedí pán na blátě,\\
v~zeleném kabátě.\\[1 mm]
{\sl Žába}\\

\noindent
Sedí panna v~sítí,\\
oči se jí svítí.\\[1 mm]
{\sl Žába}\\

\noindent
Visí, visí, visatec,\\
pod ním sedí chlupatec.\\
Až visatec upadne,\\
chlupatec ho popadne.\\[1 mm]
{\sl Žalud a divočák}\\

\noindent
Táta vysoký, máma široká,\\
syn divoký, dcera hluboká.\\[1 mm]
{\sl Živly (vzduch, země, oheň, voda)}\\

\end{multicols}
\clearpage

% End of file
