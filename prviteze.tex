% Written by Petr Gotthard
% Codepage ISO-8859-2

\section{Víte, že?}
\begin{multicols}{\value{columnsgames}columnsgames}

\subsection{Botanika}

\begin{itemize}
\itemsep -3pt

\item[-] Bambus je nejrychleji rostoucí rostlinou na světě. I když 
patří mezi traviny, chová se zcela jinak než jeho příbuzní. Ze 
země vyrůstá již zcela vyvinutý a zpočátku si to žene k~obloze 
rychlostí, kterou prý dosud žádná rostlina nepřekonala: za 24 
hodin dokáže vyrůst o více než jeden metr. Dorůstá výšky až 40 
metrů a dožívá se čtyřiceti až šedesáti let. Bambusy se používají 
jako stavební materiál, topivo, jako materiál k~výrobě 
dýmek, lan, hudebních nástrojů a jako krmivo pro dobytek.

\item[-] Lesy pokrývají 29\% zemského povrchu (1970).

\item[-] Nejlehčí dřevo, které známe, je balsa z~Jižní Ameriky. 
Je pojmenováno podle indiánského plavidla, které používali Indiáni 
v~Ecuadoru. Člun zvaný balsa byl postaven ze dřeva stromu 
Ochroma loqopus.

\item[-] Nejmenší semínko z~našich stromů má bříza. Do jednoho 
kilogramu jich je zapotřebí 6.600.000

\item[-] Osika a jeřáb se dožívají 150 let, babyka 200, habr a jasan 250, 
javor mléč a borovice až 400 let. Smrk, lípa, jedle a buk jsou 
schopny dožít až tisíce roků, dub až 2000 a tis 3000 let.

\item[-] V~povodí řeky Amazonky roste mléčný strom zvaný Brosimum. 
Nařízneme-li jeho kůru, vytéká z~řezu hustá bílá šťáva. 
Připomíná kravské mléko nejen vzhledem, ale i chutí a vůní. Je 
však poněkud trpká.

\item[-] V~říši rostlin se vyskytují všechny barvy, a to v~nejrozmanitějších 
odstínech, kromě černé. Žádná rostlina nemá černou barvu.

\end{itemize}

\subsection{Člověk}

\begin{itemize}
\itemsep -3pt

\item[-] Lidské tělo obsahuje 206 kostí a 639 svalů.

\item[-] Z celkové hmotnosti těla připadá 16\% na kůži, 40\% na svaly,
25\% na kosti a 2\% na mozek.

\item[-] Srdce bije průměrně 70$\times$ za minutu a 100 000$\times$ za
den.

\item[-] V mozku je asi 10 milionů nervových buněk, z 80\% ho tvoří voda.

\item[-] Celé tělo obsahuje 70\% vody.

\item[-] Kůže průměrného člověka má plochu 2,5m$^2$ a hmotnost 2,5kg.

\end{itemize}

\subsection{Ostatní}

\begin{itemize}
\itemsep -3pt

\item[-] Kuriózní hodiny jsou na radnici bývalého pražského židovského 
ghetta. Jejich ručičky se pohybují obráceným směrem tak, jak 
se čte hebrejské písmo. Písmena z~této abecedy zároveň 
nahrazují číslice.

\item[-] Nejmenší tištěná kniha je velká 3,5 $\times$ 3,5 mm. Je psána
v~jazyce německém, italském, anglickém, francouzském, holandském, švédském
a ruském. Byla vydána v~Holandsku.

\item[-] Slovo ,,ahoj'' pochází z~latinského ,,Ad Honore Jesum'' 
(ku slávě Ježíšově). Jde o námořnický pozdrav, který se psával 
na přídě lodí.

\item[-] Světlo vodou neprochází snadno, a proto se barvy podle hloubky 
mohou měnit. Ve 25ti metrové hloubce se krev nezdá být červená, 
ale zelená.

\end{itemize}

\subsection{Zeměpis}

\begin{itemize}
\itemsep -3pt

\item[-] Amazonka je nejmohutnější řeka světa.

\item[-] Bazilika sv. Petra je chrám pro 50 000 věřících a byla navržena 
ne jedním, ale hned deseti génii renesance. Mezi nimi byl také 
Rafaelo a Michelangelo.

\item[-] Brno k 1.5.1997 mělo 48 katastrálních území, více jak 1630 ulic, 
42 000 domů, s~celkovým počtem 387 986 trvale bydlících 
obyvatel.

\item[-] Duha jako uzavřený kruh (nikoli tedy jako známý půlkruh), se 
nám jeví, letíme-li letadlem. Když stojíme na zemi, vidíme 
jen její polovinu.

\item[-] Galapágy jsou skupina ostrovů sopečného původu, ležící v~Tichém 
oceánu přesně na rovníku. Je zde 40\% druhů místních rostlin, 
které patří mezi endemity, což znamená, že se nevyskytují nikde 
jinde na světě.

\item[-] Mořské vlny dosahují výše až 12 metrů. při bouři, zemětřesení 
a podmořské činnosti sopek mohou být až 4$\times$ větší.

\item[-] Mount Everest je dnes (1997) vyšší, než v~roce 1953, 
kdy jej zdolali Sir Edmund Hillay a Šerpa Tenzing.

\item[-] Na Eiffelovu věž vede 1671 schodů.

\item[-] Podle definice rozumíme pouští takovou oblast, která má celoročně 
příjem srážek nižší než 25 cm.

\item[-] V~Antarktidě je tak málo potravy, že vše musí být využito. 
Proto vlk polární sežere zajíce sněžného i s~kůží, chlupy, 
tukem a kostmi.

\end{itemize}

\subsection{Zoologie}

\begin{itemize}
\itemsep -3pt

\item[-] Americký chřestýš na sebe upozorňuje přívěsy na konci svého 
ocasu. Jsou to zrohovatělé články. Při každém svlékání kůže, 
ke kterému dochází 3$\times$ do roka, se vytvoří jeden článek. Největší
chřestidlo se skládá z 29 článků (téměř desetiletý chřestýš).

\item[-] Bezkonkurenčním výkonem srdce se může pochlubit netopýr. V~okamžiku 
největšího vypětí jeho srdce vykoná až 800 úderů za minutu, zatímco 
v~době zimního spánku to je pouhých 16 úderů za minutu, 
což je padesátkrát méně.

\item[-] Delfín může zůstat pod vodou 15 minut. Ve spánku lehají samice 
na hladinu vody a mají dýchací otvor nad vodou. Samci spí přímo 
pod hladinou a čas od času se vynořují.

\item[-] Dospělý lední medvěd je schopen větřit mrtvou velrybu na vzdálenost 
30 km.

\item[-] Had dokáže spolknout kořist, která je na jeho rozměry až neuvěřitelně 
velká. Polyká ji v~celku. Toho dosáhne tím, že se mu při 
polykání vykloubí čelist.

\item[-] Klokan rudý může dosáhnout rychlosti až 65 km/hod a jeden skok 
měří až 12 metrů. Mládě vážící pouhý gram samo přelézá do vaku, 
kde zůstává asi 70 dní. Hned jakmile samice mládě porodí (po 
48 hodinách), může znovu zabřeznout. Může tedy mít do roka i 
více mláďat.

\item[-] Lví samci se o mláďata vůbec nestarají. Někteří je i požírají.

\item[-] Medvídek Koala vůbec nepije. Vodu získává z~blahovičníkových 
listů. Koala v~řeči australských domorodců znamená ,,žádná 
voda''.

\item[-] Mládě ledního medvěda je velikosti krysy a váží pouhých 
450 -- 900 gramů.

\item[-] Modrá velryba (jinak též plejtvák obrovský), největší živočich 
na zemi, vydá nejhlasitější zvuk (188 decibelů, tj. asi jako 
raketa při startu), který byl zaznamenán na vzdálenost 850 km. 
V~roce 1909 byla Jižní Georgii ulovena samice 33,58 metrů 
dlouhá, jejíž hmotnost přesahovala 200 tun (srdce vážilo 700 
kg). Plejtvák rovněž nejrychleji roste. Ze zárodku vážícího několik 
gramů se za 11 měsíců vyvine novorozenec, který měří kolem 7 
metrů a váží přes 2 tuny. Během kojení za den vypije 200 litrů 
mléka a přibere 90 kg. Březost trvá 340 -- 360 dní.

\item[-] Na jednom ježkovi je až 500 blech.

\item[-] Nejdéle ze všech zvířat žijí želvy. V~nezměněném stavu 
existuji již 150 milionů let. Želva obrovská, jenž roku 1918 
nešťastnou náhodou zemřela v~dělostřeleckých kasárnách 
v~Port Louis, žila v~zajetí již 152 let.

\item[-] Nejjemnější čich má sameček můry Eudia pavonia, který rozezná 
pach samičky na 11 km.

\item[-] Nejrozšířenější jsou závody hlemýžďů. Vítěz jednoho závodu ulezl 
trať dlouhou 51 cm za 5 minut a 22 sekund.

\item[-] Nejrychlejším běžcem je gepard. Má závratnou akceleraci. Maximální 
rychlosti 100 km/h dosáhne za necelé dvě sekundy.

\item[-] Nejtěžší ptačí hnízdo vybudoval orel Halliaeetus Ieucocephalus 
roku 1963 blízko St. Petersburgu na Floridě. Široké bylo 2,9m, 
dlouhé 6 metrů a vážilo 3 tuny.

\item[-] Největší hlodavec na světě, kapybara, je dlouhý až 1,4 metrů 
a váží 110 kg.

\item[-] Největší zaznamenaná hmotnost tygra usurského byla 384 kg.

\item[-] Největší zaznamenané rozpětí křídel měl albatros stěhovavý
(3,63 m).

\item[-] Největším létajícím ptákem je kondor. Při rozpětí křídel měří 
asi 3 metry.

\item[-] Největším ptákem dneška je pštros. Když vztyčí hlavu na svém 
dlouhém krku, měří až 3 metry. Pštros, ač pták, nelétá.

\item[-] Největším ptákem, který kdy žil na zemi, byl obyvatel Madagaskaru, 
obrovský pštros Aepiorius. Byl asi tři metry vysoký, vážil více 
než 500 kilogramů a snášel vejce o průměru jednoho metru.

\item[-] Největším žijícím krabem na světě je krab obrovský. Žije v~hlubinách 
u pohoří Japonska. Jeho krunýř má průměrně 30cm a rozpětí klepet 
měří přes 2 metry.

\item[-] Nejvíce nohou nemá stonožka, ale mnohonožka. Kalifornský 
druh Illacme plenipes jich má 750.

\item[-] Nejvíce zapáchající tvor, zorila velká, je cítit až na vzdálenost 
1,6 km.

\item[-] Nejvyšší změřená žirafa camelopardis dosahovala výšky 6,1 metrů. 
Její krk má však také 7 obratlů jako člověk.

\item[-] Novorozenec medvídka Koaly je velký jako fazole a váží 0,3g. 
Žije podobně jako klokan ve vaku.

\item[-] Označení ,,mustang'' vzniklo ze španělského ,,mesteno'' (,,bez 
majitele'', ,,zatoulaný''), které bylo zase odvozeno z ,,la mesta'' 
= ,,patřící každému a nikomu''.

\item[-] Paúhoř elektrický je nejnebezpečnější ze všech elektrických ryb. 
Dokáže vytvořit elektrický výboj o síle 500 -- 600 voltů. Hlava 
působí jako kladný pól, ocas jako záporný.

\item[-] První inkoust na psaní se vyráběl z~barviva, které se 
tvoří v~inkoustovém vaku chobotnic.

\item[-] Ptáci mají nejméně 8x ostřejší zrak než lidé. Sokol stěhovavý 
může zpozorovat holuba na 8 km.

\item[-] Slon africký, největší suchozemský živočich, sní za jeden den 
225kg rostlin a vypije 136 litrů vody najednou. Největší exemplář 
měřil v~kohoutku 4,16 m, délku měl 10,67 metrů a vážil 
12 tun. Nejdelší sloní kel měří 3,49 metru.

\item[-] Sova dokáže během roku vyhubit až tisíc polních či lesních myší.

\item[-] Šimpanz je jediné zvíře, které se pozná v~zrcadle.

\item[-] V ,,pásmu ozvěn'' v~hloubce 600 -- 1200 metrů pod hladinou 
se keporkakové (druh velryby) mohou slyšet z~jedné strany 
Tichého oceánu na druhou.

\item[-] V~České republice žije na 40000 různých živočichů (velký 
počet je hmyzu). Toto číslo tvoří pouze půl procenta ze všech 
živočišných druhů světa.

\item[-] V~závodě dešťovek (žížal) zdolala soutěžící jménem Willie 
trať dlouhou 60cm za 2 minuty a 15 sekund.

\item[-] Včela na své cestě z~úlu za medem, která trvá vždy přibližně 
10 minut, vysaje šťávu z~asi sta květů. Je-li teplo a sucho 
vykoná za den asi 30 až 40 takových letů.

\item[-] Velké kudlanky napadají a požírají žáby a ptáčata.

\item[-] Vrcholová rychlost lenochoda (který prospí 18 hodin denně) je 
1 km/h.

\item[-] Žralok lidožravý vycítí jednu kapku krve i v 4,6 milionech litrů 
vody.

\end{itemize}

\end{multicols}
\clearpage

% End of file
