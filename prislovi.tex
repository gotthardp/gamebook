\section{Staročeská přísloví}
\begin{multicols}{\value{columnsgames}}

\noindent
{\large\bf Bajky jednou větou}\\[1 mm]
Ač krtek pod zemi chodí, přece se ukrýti nemůže.\\
Běda tomu domu, kde tele rozkazuje volu.\\
Býval volem, a nyní nechce zůstat ani koněm.\\
Co se z kočky narodí, nebude než myši chytati.\\
Čím se koza pyšní, za to se ovce stydí.\\
Datel dřevo razí, sám sobe nos kazí.\\
Každá liška svůj ohon chválí.\\
Kdo honí dva zajíce, nechytí žádného.\\
Kdyby byl zajíc nespal u Malina, byl by už doběhl do Kolína.\\
Kdyby bylo po vůli psin nezůstalo by kobyly ve vsi.\\
Když husa za moře zaletí, přiletíc přec husou zůstává.\\
Když se káně zjestřábí, více škrábe, než rodilý jestřáb.\\
Kočičí hra - myší smrt.\\
Kůň, který oves dobývá, nejmíň se ho nají.\\
Mladého raka káral starý rak, proč leze pořád znak. Neukázal
mu lépe jak, i lezou oba podnes tak.\\
Mnoho psů - zajícova smrt.\\
Myš jest malé zvíře, přec nevěří jedné díře.\\
Neleží v tom sila, kobyla-li sivá, ale jak táhne.\\
Nelítejte, vrány, na vysoké chrámy.\\
Nešťastný taky dům, kde kohout mlčí a slepice zpívá.\\
Nežeň se, psíku, po vlčí stopě; ohlédne se a sežere tě!\\
Osel oslu pěknější arabského koně.\\
Oslu když se dobře vede, jde na led tancovat.\\
Pes dvojích vrat má často hlad.\\
Po tmě je každá kráva černá.\\
Potrefena husa z hejna kejhá.\\
Psi jedné vsi často se hryžou, ale na vlka spolu se derou.\\
Rad kocour ryby jídá, ale nerad pro ně do vody břede.\\
Slepice kdákajíc, zrna v zobáku neudrží.\\
Sobe vlk ostronos, sobe ježek kadeřavým.\\
Svine, když kaliště mají, na cistou vodu nedbají.\\
Sysel, dosti chytré zvíře, prozradí se přece v díře.\\
Ta myška už lapena s kterou kočka si pohrává.\\
Také černá kráva bílé mléko dává.\\
Tím řeka horší není, ze z ní psi pijí.\\
Unavenému koni je i ocas těžký.\\
Už vejce moudřejší než slepice?\\
V nerybných krajích i rak za rybu platí.\\
Viděla žába koně kout, i zdvihla též nohu.\\
Vinen medvěd, že krávu snědl, vina i kráva, že do lesa šla.\\
Vše domu přijde, co vlci nesežerou.\\
Vůl starší učí orati vola mladšího.\\
Zajíc dlouhé uši má a není oslem. Kozel dlouhou bradu má a není
mudrcem.\\
Zle koni, kolem něhož osli hýkají.\\
Žádná kráva není, aby telátkem nebyla.\\

\noindent
{\large\bf Čertovinky}\\[1 mm]
Čert má mnoho cukru, a proto hřích činí sladkým.\\
Čert vždycky na větší hromadu klade.\\
Čiň čertu dobře a peklem se ti odmění.\\
Dobře je i v pekle mít přítele.\\
Kde ďábel nic nezmůže, aspoň zasmradí.\\
Není toho kostelíčka, aby čert při něm svou kapličku neměl.\\
Nepiš čerta na stěně, sám se on maluje.\\
Nevíš, co stojí mezi anděly a čerty? Žerty!\\
V nouzi čert i mouchy lapá.\\

\noindent
{\large\bf Jednou hostem, podruhé hostitelem}\\[1 mm]
Až se najím, půjčím ti lžíci.\\
Co se snese v lese, nesnese se na plese!\\
Host první den zlato, druhý stříbro a třetí med. Honem domu jeď!\\
Host se zuby jen a s panděrem, buď na osla posazen!\\
Kde jedí, tam jez, kde pracují, tam nepřekážej.\\
Kde nescházím, nechci zbývati.\\
Kdo se mnou chléb jíst nechce, s tím já koláče nebudu.\\
Napřed já budu jísti a ty se dívej, potom zas ty budeš se dívat 
a já budu jísti.\\
Neohryzuj kosti, nechej něco i pro hosty!\\
Při hostech lépe je za tolar škody, než za halíř ostudy.\\
Přijde kmotr na oběd, a už lžíce nebude.\\
Vidě jámu, v ní se nekoť, nezván pak na hody nechoď!\\

\noindent
{\large\bf Než zasedneš ke stolu}\\[1 mm]
Čím oko napaseš, tím břicho nenaplníš.\\
Hladové břicho nestyda a syté ještě větší.\\
Kdo chce spáti sladce, ať večeří krátce.\\
Kdo naložil do břicha, rád se jazykem potýká.\\
Když jí psy bohatý, řekne se, že to dělá pro své zdraví; když 
jí psy chudý, tvrdí se, že je mrchožrout.\\
Máslo před obědem zlato, po obědě stříbro, po večeři olovo.\\
Najíš-li se slaniny, nemůžeš být už večer tlustší.\\
Snídej sám, o oběd se rozděl s přítelem a večeři nechej svému 
nepříteli.\\
Sytá-li myš, horká ji mouka.\\
Víno neříká jdi, ale seď!\\

\noindent
{\large\bf Nic lidského nám není cizí}\\[1 mm]
Budou-li mě dobrým bít, neuvěřím, že je dobré.\\
Cizí huba není chlév. Nemůžeš ji zavřít, ani když páchne.\\
Co je dovoleno pánu, není dovoleno kmánu.\\
Co platno koření, když není vaření.\\
Co se rychle vznítí, nedlouho svítí.\\
Do řídkého bláta kamenem neházej!\\
Dokud byl kratší v hodnosti, byl o loket delší v moudrosti.\\
Dřív jen okolo bot chodíval, a teď neví, jak si v nich vykračovat.\\
Hadr onuci tresce, žádný se polepšit nechce.\\
Hloupý říká, co ví; moudrý ví, co říká.\\
Hrnec hrnci kaze, oba černí jako saze.\\
Hustá přísaha, řídká pravda.\\
Chválí cizí krajiny, a sám ani krok z dědiny.\\
I vlas má svůj stín (chybu).\\
Jeden blázen může víc otázek nadělati, než mu deset mudrců stačí 
odpovídati.\\
Kámen často hýbaný mechem neobroste.\\
Kdo chodí v zápasy, ať nepláče o vlasy.\\
Kdo na dvou stolicích sedí, snadno mezi nimi na zem spadne.\\
Kdo po zemi chodí a do nebe hledí, snadno se uhodí.\\
Kdo se chlubí, čest svou hubí.\\
Kdo tone, sekyru slibuje. A když vytáhneš, i topůrka líto.\\
Koho chleba jíš, toho píseň zpívej.\\
Lepší o nás dobrý hlas, než ze zlata pás.\\
Meč na mrtvém nezkoušej.\\
Moudrý se umí i od hloupého poučit.\\
Na hlavě šperky, a v hlavě čemerky.\\
Nebuď do každého hrnce vařečka!\\
Nekoukej na to, abys byl lepší než druhý, ale abys byl lepší 
než včera.\\
Nelez tam, tvářičko, kam neproleze hlavička.\\
Němý hluchému pošeptal, aby se slepý podíval, jak beznohý utíká.\\
Netřeba se třtině báti, větrové kdy dub vyvrátí.\\
Po lidských žlabech kalná voda teče.\\
Pomlouvat nepřítomného je jako bít mrtvého.\\
Prodej náušnice a kup si hřeben.\\
Snáze projdeš tri vsi lačný, nežli jednu nahý.\\
Sud nesvárný, i co do něho vlije, znečistí.\\
Svrbný drbného vždy najde.\\
Víc se nenatahuj, než se můžeš přikrýt.\\
Z cizí kůže se dobře široký řemen krájí.\\

\noindent
{\large\bf O dárcích a podarovaných}\\[1 mm]
Darovati jest pansky - brat nazpět cikánsky.\\
Hezká peřinka, škoda jen, že rukávů nemá.\\
Kdo chce komu dáti, nemá se ho ptáti.\\
Kdo chce vejce míti, musí si kdákání nechat líbit.\\
Kdo neděkuje za málo, nepoděkuje ani za mnoho.\\
Krajíc k bochníku víc nepřistávej!\\
Lepší dnes vejce, než zítra slepice.\\
Z obce po nitce a nahý má na košili.\\

\noindent
{\large\bf O penězích a hospodaření}\\[1 mm]
Boháč želí korábu a žebrák mošny.\\
Cizí jmění sebevětší - netrvale; lepší spravedlivě, třeba malé.\\
Co jeden hloupý koupí, ani sto rozumných neprodá.\\
Dobrý kůň i ve stáji kupce najde.\\
Dokud prosí, zlata slova nosí, před plácí záda obrací.\\
Dokud u mne nacházel, i v noci přicházel; a teď, co jsem chudý, 
neví ani ve dne kudy.\\
Kazdy trhan říká: Kdybyste věděli, jací jsou mí strýčkové páni!\\
Kdo lituje podkovničku, zmrhá podkovu; kdo lituje podkovy, zmrhá 
koně; a kdo lituje koně, zmrhá sám sebe.\\
Kdo myslí na sebe, když se má dobře, zůstane sám, upadne-li do 
bídy.\\
Kdo neumí střádati, musí často strádati.\\
Kdo pustí hlad do břicha pro groš, ani za dva ho nevyžene.\\
Kdo za cizí vlnou odchází, sám ostřižen domů přichází.\\
Kdo želí malé výlohy, učí se pozdě ze škody.\\
Kdybychom měli tolik mouky, kolik nemáme másla, napekli bychom 
buchet pro celou ves.\\
Když hospodský na křídu dává, to mnohý skáče. Ale maje platit 
ošívá se a pláče.\\
Kup, čeho nepotřebuješ, a brzo poznáš, bez čeho nelze.\\
Lakomci nikdy dost není, i kdyby měl všeho světa jmění.\\
Lakomství a oheň, přikládaje neuhasíš.\\
Lepe státi za svým snopem, než za cizím mandelem ležet.\\
Lepší svůj šat plátěný, než hedvábný kradený.\\
Málokdo je tak bohatý, aby mohl kupovat laciné věci.\\
Mnohý člověk považuje za bratra svou kapsu.\\
Ne hned nový dům vystaví, kdo starý zboří.\\
Nedbám o malý hrnec, když se z malého najím.\\
Nepůjde ti k duhu, dáš-li jísti s sebou dluhu.\\
Nevidomky jen vejce se kupují.\\
Peníze otevírají svět, ale zavírají srdce.\\
Sam sobe troufej, v cizí kapsu nedoufej!\\
Skoupý člověk jako jeho skříně - ač v nich zlato, přece v koutě
stojí.\\
Špatná hospodyně bývá, kterou slunce v duchnách vídá.\\
Víc škoda střevíce než nohy, říká lakomec ubohý.\\
Všude hrabe, nikde vidle.\\
Z laciné koupi raduje se hloupý.\\
Z pilnosti se štěstí rodí, lenost holou bídu plodí.\\
Zahálky jsa služebníkem, neběduj, žes hadrníkem.\\
Ztracené peníze se ti jednou vrátí, ale ztracená čest nikdy.\\

\noindent
{\large\bf O práci a učení}\\[1 mm]
Ani les neposekej, ani bez dříví domu nechoď.\\
Co pronikneš, tím vynikneš.\\
Čím dále v lese, tím více dřev.\\
Drž se ševče svého kopyta!\\
Dříve dvě ruce z vozu sházejí než deset naloží.\\
Hledě na les nevyrosteš, hledě na práci nic neuděláš.\\
Já dáma, ty dáma, voda se přinese sama?\\
Jaké na mlýn nasypeš, takové se semele.\\
K čemu kdo chuť má, v tom těžkosti nepozná.\\
Kdo nerozumí kování, nechť mi kladivo nehaní.\\
Kdo se bojí vrabců, ať raději neseje.\\
Kdo táhne, toho pohánějí.\\
Když holka neumí tancovat, nadává, že zem je hrbolatá.\\
Když vůz namažeš, jako bys třetího koně připřáhl.\\
Kovář štěstí nekuje, každý je sobe hotuje.\\
Lepší je s moudrým roztloukat kamení, nežli jíst buchty s hlupákem.\\
Lepší znání s pochybou, než neznání s oblibou.\\
Mlať, dokud se mlátí, mluv, dokud tě poslouchají.\\
Nechval ženino tílko, ale její dílko.\\
Nemůžeš-li mi dát podpory, nečiň mi závory.\\
Povídá strom sekeře. Jakpak bys mě porazila, kdybys nemela ze 
mě topůrko?\\
Pro zimu nechtěl v létě orati, nedávej mu až bude v létě žebrati.\\
Proč letí pták k lesu? Protože les k ptáku nepřiletí.\\
Prudký se tolikrát překotí, až ho i váhavý dohoní.\\
Přehnala královna práci, až ji prst zabolel.\\
Řekni teplo kolikrát chceš, teplo nebude, dokud oheň nerozděláš.\\
Skoupý haní cizí hostiny, a přece na ně chodí.\\
Šlépěje hospodářovy pole tučným činí.\\
Švec dokud jednu botu neušije, druhou nezačíná.\\
V lete chystej sáně a vůz v zimě.\\
Vodu, která uplynula, nelze pustit na mlýn.\\
Z malého hocha statný synek bývá, mladý stromek lehce se ohýbá.\\

\noindent
{\large\bf Pravdy věčně živé}\\[1 mm]
Cenu věcí, které máme, poznáme, až je ztratíme.\\
Cizí vidíš pod lesem a své nevidíš pod nosem.\\
Čím kolo u káry horší, tím víc vrže.\\
Člověk jde za štěstím a za člověkem neštěstí se žene.\\
I křivé dříví dobře hoří.\\
Jak velké hodiny ukazují, tak malé se po nich opravují.\\
Každého jednou jeho štěstí navštíví, jenže ne každý ví, co jeho 
štěstí je.\\
Kde jednou smetiště, lidé víc naházejí.\\
Kdo čemu chce, dá se jedním vlasem přitáhnout.\\
Kdo má desatery housle, nemusí být ještě dobrý hudebník.\\
Kdo se chce kulhavému smáti, musí sám rovně státi.\\
Kdo za dveřmi poslouchá, sám o sobě slýchá.\\
Když se dva prohánějí, třetímu nadhánějí.\\
Koho had uštkl, i stočeného provazu se bojí.\\
Koho kaše spálila, ten i na podmáslí fouká.\\
Komu bradu holí, nemůže po vůli mluviti.\\
Lepší doma krajíc chleba, než v cizině kráva celá.\\
Lepší hrst jistoty, nežli pytel nadějí.\\
Lež, ačkoli snídá, zřídkakdy obědvá a nikdy téměř nevečeří.\\
Lidé často říkají, co si z prstu vysají.\\
Mám-li se kořiti křápu, tedy raději škorni.\\
Na chválené hrozny ber malý košík.\\
Na jednu loď neklaď všechno své zboží.\\
Ne každý, kdo jazykem seče, se hodí do bitvy.\\
Nepouštěj se na moře bez vesla.\\
Nepřilévej louže k blátu.\\
Neslouží útěk ke cti, ale ke zdraví.\\
Nestahuj kalhoty, když daleko ještě brod.\\
Ošklivá tvář zrcadla nemiluje.\\
Prázdný sud nejvíc duní.\\
Pro třísku v oku mém, břevno ve svém vlastním nevidíš.\\
Říkej pravdu, až víš, kudy utečeš.\\
Sám sobě hudeš, sám vesel budeš.\\
Slavný plavec nejednou v malé říčce utone a slavného rytíře v 
hospůdce zabijí.\\
Směj se, a lidé se budou smát s~tebou. Plač, a plakat 
budeš sám.\\
Staré chrámy dobré zvony mají (a staří lidé radu).\\
Sůl na chléb, a ne chléb na sůl se dává.\\
V lese dříví sekají a do vsi třísky létají.\\
Vlídnou řečí více získáš, nežli sečí.\\
Zloděj u lháře rád hospodou stává.\\

\noindent
{\large\bf Přítel, nebo nepřítel?}\\[1 mm]
Dobrá studna v suchu vodu dává, dobrý přítel v nouzi se poznává.\\
Hedvábnou rukou přítele vybírej a železnou drž!\\
Hladové srdce je horší než hladová střeva.\\
Horší jazyk falešníka, nežli kopí bojovníka.\\
Kdo komu miloučký, i neumyt běloučký.\\
Kdo se chce zalíbit každému, nezalíbí se žádnému.\\
Láska je jako slza - rodí se v očích a padá k srdci.\\
Lepší blízký soused, než daleký přítel.\\
Ne ten, kdo milý, kdo krásný, ale ten krásný, kdo milý.\\
Přítel příteli hrad staví, nepřítel nepříteli rakev teše.\\
Přítele tajně napomínej a zjevně chval.\\
Rána po meči se zahojí, rána po zlém slovu nepřestane bolet.\\
Špatný hudebník zkazí dobré housle, dobrého člověka zkazí špatní 
přátelé.\\
Věř, ale komu věříš, měř!\\

\noindent
{\large\bf Rodina a příbuzní}\\[1 mm]
Děvče do dvanácti češ, do šestnácti střež a po šestnácti 
děkuj tomu, kdo vybere dceru z domu!\\
Jablko nepadá daleko od stromu.\\
Jaký otec, taký syn, jaká voda, taký mlýn, jaké dřevo, taký klín.\\
Jazyk dobře uvěří, když zub zabolí.\\
Jeden otec snáze deset dětí uživí, než deset dětí jednoho otce.\\
Na otci vodu nosívali, tak synovi nehoď s chomoutem na oči.\\
Nekazí děti hračky, ale zlý příklad.\\
Nos nemůže uťat býti, aby ústa nebyla zkrvavena.\\
Svěř se mu se svou radosti, bude dvakrát větší.\\
Svěř se svému bratrovi se svou bolestí, zmenší se na polovic.\\

\noindent
{\large\bf Zvířata člověku zrcadlem}\\[1 mm]
Ať děláš, co děláš, psu ocas nenarovnáš.\\
Čí jalovice, toho i provaz.\\
Dej krávě do držky, ona ti dá do dížky.\\
Falešný přítel je jako kočka. Zepředu líže, zezadu škrábe.\\
Hovnivál nic jiného nezná, než se v trusu rýpat.\\
Chlub se baba strujem a kráva dujem.\\
I kdybys svému psu nohu uťal, on za tebou poběží.\\
Jest věru věc těžká, pěstí zabít ježka.\\
Ještě vlka nezabili, a už na jeho kůži pili.\\
Kdo chce psa bít, hůl si vždycky najde.\\
Kdo lehá mezi otruby, přichází svini pod zuby.\\
Kdo se psy lehá, ten s blechami vstává.\\
Kdyby osla do Paříže vedl, komoně z něho neudělá.\\
Když pět lidí bude říkat, ze vůl je kráva, říkej to také. Ale
nechoď ho dojit!\\
Když přijdeš mezi vrány, musíš krákat jako ony.\\
Když ptáčka lapají, pěkně mu zpívají.\\
Když už mám jet do pekla, tak na dobrém koni.\\
Kolik děr, tolik syslů, tolik hlav, tolik smyslů.\\
Krásný páv peřím a člověk učením.\\
Krm vlka jak chceš, on vždy k lesu hledí.\\
Kůň jest jen jednou hříbětem, ale člověk je dvakrát dítětem.\\
Lepší jeden pták na talíři pečený, nežli dva v povětří vznesení.\\
Mám-li spadnout, tedy z dobrého koně.\\
Mířil na vránu a trefil krávu.\\
Myš do díry nemohla, tykev nesla.\\
Ne proto vlka bijí, že je šerý, ale že ovci snědl.\\
Ne toho pták, kdo ho škube, ale kdo ho jí.\\
Nebere se z jednoho vola dvou kůží.\\
Nežli slepého koně voditi, lépe jest pěšky choditi.\\
Pan nemůže pro kord a pes pro ocas (dveře zavřít).\\
Panská láska po zajících skáče.\\
Poruč psu, pes poručí ocasu. Pes si lehne a ocas se ani nehne.\\
Pozdě zajíce za ocas chytat, když jsi ho nechytil za uši.\\
Právo k pavučině se srovnává. Brouk ji prorazí, moucha zůstává.\\
Sousedova kráva více mléka dává.\\
Strč na svini i zlatohlav, přece sviní zůstane.\\
Vrána zůstane černá, i kdybys ji ve sněhu vyválel.\\
Zrnko po zrnku kuřátko naplní svoje volátko.\\

\end{multicols}
\clearpage
