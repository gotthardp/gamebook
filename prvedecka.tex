% Generated 24.8.2000 16:55:08
% Codepage ISO-8859-2

\section{Vědecká přísloví}
\begin{multicols}{\value{columnsgames}}

\noindent
Navštěvovati podnik veřejného stravování za stavu insuficietního
finančního zabezpečení se zapovídá.\\[1 mm]
{\sl Bez peněz do hospody nelez.}\\

\noindent
Jestliže není konána činnost, která polidštila opici, není
možno očekávati cukrářské pečivo.\\[1 mm]
{\sl Bez práce nejsou koláče.}\\

\noindent
Při poklesu produktivity práce na nulu projeví se totální
nedostatek kruhového pečiva, působícího obezitu obyvatelstva.\\[1 mm]
{\sl Bez práce nejsou koláče.}\\

\noindent
U závodu na zpracování obilí, řízených mytologickými bytostmi
je poměrně nízká produktivita práce vyvážena naprostou
spolehlivostí.\\[1 mm]
{\sl Boží mlýny melou pomalu, ale jistě.}\\

\noindent
Verbální komunikace, při které poklesne intenzita signálu na minimum
slyšitelné pouze v bezprostředním okolí zdroje, je v přímé korelaci
s jednoduchým strojem na vyorávání brambor.\\[1 mm]
{\sl Co je šeptem, to je s čertem.}\\

\noindent
Činnost (vykonání činnosti) nemůže osoba (dále jen proveditel)
provést v období, tedy časovém úseku, které by bylo delší než 1 (jeden)
den ode dne, kdy jsou splněny všechny předpoklady pro vykonání této
(výše uvedené) činnosti proveditelem.\\[1 mm]
{\sl Co můžeš udělat dnes, neodkálej na zítra.}\\

\noindent
Poznatky získané cílevědomě v preadolescentním věku jsou adekvátní
poznatkům získaným náhodně ve věku seniorském.\\[1 mm]
{\sl Co se v mládí naučíš, ve stáří jako když najdeš.}\\

\noindent
Kategoricky imperativ nedovoluje libovolnému subjektu likvidovat
zdroj infrazáření, který svým behaviouristickým systémem neatakuje
týž subjekt, neboť jeho radiace se chová inertně k tělesně integritě
dotyčného subjektu.\\[1 mm]
{\sl Co tě nepálí, nehas.}\\

\noindent
Všudypřítomná objektivní filosofická kategorie označující
speciální fyzikální dimenzi vztahující se k behaviorálním změnám
jsoucna je totéž jako blíže neurčené množství monetárních jednotek.\\[1 mm]
{\sl Čas jsou peníze.}\\

\noindent
Působení v souladu se záměry mytologické bytosti žijící v
místech o teplotě, jež je statisticky hodnocena jako nadprůměrně
vysoká, implikuje transport tamtéž.\\[1 mm]
{\sl Čiň čertu dobře, peklem se ti odmění.}\\

\noindent
Donace individua rodu Equus neopravňuje akceptora k vizuální
percepci dentální soustavy tohoto jedince.\\[1 mm]
{\sl Darovanému koni na zuby nehleď.}\\

\noindent
Prováděti dentální inspekci donovaného živočicha rodu Equus
je nepřípustné.\\[1 mm]
{\sl Darovanému koni na zuby nehleď.}\\

\noindent
Druhá mocnina tří profesí se jeví být prvním dvojčíslím dekadické
soustavy pauperizace.\\[1 mm]
{\sl Devatero řemesel --- desátá bída.}\\

\noindent
Prostorová diskontinuita hmotného tělesa neovlivní časovou
kontinuitu procesu nanášení laku na jeho povrch.\\[1 mm]
{\sl Díra --- nedíra, furt se natírá.}\\

\noindent
Kvalifikovaná a produktivní pracovnice v~domácnosti
za účelem získání části tělního pokryvu drobného obratlovce dokáže
uvést své tělo do pohybu po parabolické dráze, jejíž počáteční
a cílový bod leží na různých stranách bariéry oddělující dvě
různé plochy zapsané samostatně v~katastrálních mapách.\\[1 mm]
{\sl Dobrá hospodyně pro pírko i přes plot skočí.}\\

\noindent
Prognózu optimálního okamžiku pro své akce proveď podle modelového
vztahu domestikovaného vodního opeřence k plodenství kulturních
trav.\\[1 mm]
{\sl Dočkej času jako husa klasu.}\\

\noindent
Velikost fyzikální veličiny určuj dvakráte častěji, nežli používáš
způsobu obrábění, při němž je materiál dělen na části.\\[1 mm]
{\sl Dvakrát měř, jednou řež.}\\

\noindent
Více než jednou, ale méně než třikráte, urči velikost fyzikální
či chemické veličiny a méně než dvakráte, ale více než nulkrát
použij způsobu obrábění, jímž se části materiálu od sebe oddělují.\\[1 mm]
{\sl Dvakrát měř, jednou řež.}\\

\noindent
Tělesným orgánem se zpevněným skeletonem a obsahujícím nervové
řídicí centrum nelze narušit souvislost bariéry sestavené z kompaktních
částí horniny popřípadě zeminy zpracované do pravidelných kvádrů a
spojené směsí hornin s podstatným obsahem vápence.\\[1 mm]
{\sl Hlavou zeď neprorazíš.}\\

\noindent
Přemístí-li se do příbytku patřícího mně obyvatel příbytku
jiného, je radno vložiti do vlastní horní končetiny hrubě opracovanou
část stromu.\\[1 mm]
{\sl Host do domu, hůl do ruky.}\\

\noindent
Kategorický imperativ velí při volbě způsobu techniky přenosu
intelektuální pomoci směrem k jedinci schopnému samostatného
abstraktního myšlení přiklonit se k verbálnímu projevu, zatímco
u ostatních subjektů volit přímý prudký tělesný kontakt.\\[1 mm]
{\sl Chytrému napověz, hloupého kopni.}\\

\noindent
Schopnost operativně řešit životní situace k vlastnímu prospěchu
na základě vyššího stupně intelektu hominis sapiensis se nerovná
drahám libovolného tvaru, vedeným jakýmkoli bodem prostoru.\\[1 mm]
{\sl Chytrost nejsou žádné čáry.}\\

\noindent
Je-li $B$ bodem ukončení trajektorie malvice, puzené gravitační
silou z bodu $A$, bod $C$ je kolmým průmětem bodu $A$ a bod $A$ je
na stromě,
potom vzdálenost bodů $B$ a $C$ se limitně blíží
k nule.\\[1 mm]
{\sl Jablko nepadá daleko od stromu.}\\

\noindent
Velikosti horizontálních průmětů drah těles oddělených od
sporofytu krytosemenné rostliny jsou omezené.\\[1 mm]
{\sl Jablko nepadá daleko od stromu.}\\

\noindent
Vzdálenost bodu $A$ (což je místo, kde ukončí, puzena gravitační
silou, svou dráhu malvice) od bodu $B$ (což je místo ležící svisle
pod místem započetí její dráhy) se blíží k nule.\\[1 mm]
{\sl Jablko nepadá daleko od stromu.}\\

\noindent
Informační obsah mechanického vlnění vycházejícího z orálního
otvoru humanoidní bytosti a postupujícího směrem k souvislému
porostu vyšších dřevin se nezmění ani při následné změně směrového
vektoru ve vektor opačný.\\[1 mm]
{\sl Jak se do lesa volá, tak se z lesa ozývá.}\\

\noindent
Náraz akustických vln, šířících se plynným prostředím nízkých
vrstev atmosféry ze zdroje v lidském hrdle, odráží se od bariéry
kompaktního přírodního systému, přičemž síla i kvalita zpětné vlny
je adekvátní původnímu impulsu.\\[1 mm]
{\sl Jak se do lesa volá, tak se z lesa ozývá.}\\

\noindent
Kvalita počáteční fáze totální regenerace organismu subjektu
a intenzita prožitku z téhož jsou přímo úměrné úsilí, vynaloženému
subjektem k předchozí optimalizaci prostoru pro regeneraci organismu
určeného.\\[1 mm]
{\sl Jak si kdo ustele, tak si lehne.}\\

\noindent
Modus presomnatické lokalizace subjektu je definován jím
provedenou adjustací prostředí.\\[1 mm]
{\sl Jak si kdo ustele, tak si lehne.}\\

\noindent
Generace filiální přejímá četné znaky generace parentální.\\[1 mm]
{\sl Jaký otec, takový syn.}\\

\noindent
Ztráta prostorového vnímání není na závadu suverenitě nad
osobami touto ztrátou ještě více postiženými.\\[1 mm]
{\sl Jednooký mezi slepými králem.}\\

\noindent
Všichni jednotliví zástupci světové aviafauny vyzdvihují pouze
přednosti svého vlastního tělního pokryvu.\\[1 mm]
{\sl Každý pták chválí své peří.}\\

\noindent
V závodech na zpracování obilí se objednávky zpracovávají
systémem FIFO.\\[1 mm]
{\sl Kdo dřív přijde, ten dřív mele.}\\

\noindent
Pátrání po lignu adaptovaném na předmět denní potřeby se
u reflektanta ataku šelmy čeledi Canidae setkává s úspěchem za
všech podmínek.\\[1 mm]
{\sl Kdo chce psa bít, hůl si vždy najde.}\\

\noindent
Podmínkou koexistence jedince druhu Homo sapiens a společenství
druhu Canis lupus je sjednocení akustické signální soustavy.\\[1 mm]
{\sl Kdo chce s vlky žíti, musí s nimi výti.}\\

\noindent
Subjekt $A$, jenž vyvíjí úsilí o vytvořeni svislého či úklonného
díla ústícího na povrch a determinujícího subjekt $B$, sám opíše
dráhu ve zmíněném díle ústící.\\[1 mm]
{\sl Kdo jinému jámu kopá, sám do ní padá.}\\

\noindent
Občan záměrně poskytující klamné informace, současně neoprávněně
(což je axiom) převádí do svého vlastnictví majetek spoluobčanů,
popřípadě společnosti.\\[1 mm]
{\sl Kdo lže, ten krade.}\\

\noindent
Subjektu, který nemá dostatek positivního přesvědčení o pravdivosti
výroku, se striktně doporučuje provést akcelerovanou autolokomoci
směrem vedoucím k fyzické verifikaci výroku.\\[1 mm]
{\sl Kdo nevěří, ať tam běží.}\\

\noindent
Zástava encefalické činnosti jedince je zcela irelevantní k oběma
limitním stavům kvantity příjmu tekutin u téhož individua.\\[1 mm]
{\sl Kdo pil, umřel, kdo nepil, umřel taky.}\\

\noindent
Při zjištění záměrného transportu části horniny od občana
$A$ k občanu $B$ je posledně jmenovaný povinen uskutečnit zpětný
přesun po stejné dráze, tedy v opačném směru, k prvně jmenovanému,
avšak s použitím žitného pečiva.\\[1 mm]
{\sl Kdo po tobě kamenem, ty po něm chlebem.}\\

\noindent
Časová distance mezi příchodem libovolného individua a termínem
k této události vhodným implikuje sebeújmu.\\[1 mm]
{\sl Kdo pozdě chodí, sám sobě škodí.}\\

\noindent
Subjekt, mající dlouhodobý průměr první derivace dráhy v
závislosti na čase nižší než je přípustná mez, působí tímto sobě
újmu.\\[1 mm]
{\sl Kdo pozdě chodí, sám sobě škodí.}\\

\noindent
Osoby trpící stavy chorobné či oprávněné úzkosti musí respektovat
zákaz vstupu na území porostlé dřevinami.\\[1 mm]
{\sl Kdo se bojí, nesmí do lesa.}\\

\noindent
Subjektu, v jehož informační databázi je obsažena relace
nadměrné obavy, jest zapovězeno pohybovat se v biotopu s rozsáhlým
stromovým patrem.\\[1 mm]
{\sl Kdo se bojí, nesmí do lesa.}\\

\noindent
Nechť $M$ je množina osob $x$ takových, že $x$ patří do $M$, právě když
$x$ vydává nonverbální signály $S$ vyjadřující pozitivní emoci $E$. Dále nechť
$f$ je uspořádání množiny $M$ podle času vzhledem k okamžiku signálu $S$ a
$g$ nechť je dobré uspořádání množiny $M$. Potom platí:\\
Je-li $y$ maximální prvek množiny $M$ vzhledem k uspořádání $f$, potom je také
maximálním prvkem množiny $M$ vzhledem k uspořádání $g$.\\[1 mm]
{\sl Kdo se směje naposled, ten se směje nejlíp.}\\

\noindent
Produkce($x$) $>$ spotřeba($x$) $\Rightarrow x = 3$\\[1 mm]
{\sl Kdo šetří, má za tři.}\\

\noindent
Množina oblastí výskytu samců druhu Felis domestica a množina
míst charakterizovaných abundancí nekontrolované činnosti drobných
domácích hlodavců jsou navzájem disjunktní.\\[1 mm]
{\sl Když je kocour pryč, myši mají pré.}\\

\noindent
Antagonie dvou subjektů ústící až do vzájemné mechanické
interakce zaměřené na nabytí dominance tvoří kauzální nexus pro
rychlý periodický pohyb bráničního svalstva subjektu třetího,
navíc doprovázený specifickým akustickým projevem.\\[1 mm]
{\sl Když se dva perou, třetí se směje.}\\

\noindent
Realizuje-li pár antagonistických jedinců reciproční agresi,
vykonává redundantní člen triády emocionální projev regulující
vlastní psychickou tenzi.\\[1 mm]
{\sl Když se dva perou, třetí se směje.}\\

\noindent
Číslo, jímž můžeš vyjádřit svou lingvistickou potenci, se
rovná číslu, jímž znásobuješ své vlastní ego.\\[1 mm]
{\sl Kolik řečí znáš, tolikrát jsi člověkem.}\\

\noindent
Blokování svislého informačního kanálu nelze kompenzovat
finanční transakcí typu emptio venditio ve výdejně léků.\\[1 mm]
{\sl Komu není shůry dáno, v apatyce nekoupí.}\\

\noindent
Kdo aplikuje vlastní energii na vykonání práce místo jejího
pozvolného uvolňování do prostoru, bývá obklopen chlorofylem.\\[1 mm]
{\sl Komu se nelení, tomu se zelení.}\\

\noindent
Kdo odolává pokušení nepodlehnout touze nechat dřímat vlastní
energii, bývá obklopen chlorofylem.\\[1 mm]
{\sl Komu se nelení, tomu se zelení.}\\

\noindent
Samice domácího lichokopytníka ve vlastnictví podnikatele v oblasti
metalurgie je v přímém tělesném kontaktu s povrchem, po němž se
pohybuje.\\[1 mm]
{\sl Kovářova kobyla chodí bosa.}\\

\noindent
Samičí jedinec domácího lichokopytníka patřící mistru ohně
a železa se většinou po světě pohybuje neobutý.\\[1 mm]
{\sl Kovářova kobyla chodí bosa.}\\

\noindent
Kinetická energie eroticky motivovaného bilaterálního, zpravidla
heterosexuálního vztahu může být použita k transferu vysokých
geologických útvarů.\\[1 mm]
{\sl Láska hory přenáší.}\\

\noindent
Užitná hodnota malého množství proti mobilitě manuálně zajištěné
ptačí svaloviny je vyšší nežli výrazně větší množství obdobného materiálu
mobilního a obtížně dostupného.\\[1 mm]
{\sl Lepší vrabec v hrsti nežli holub na střeše.}\\

\noindent
Faktická hodnota již uchopeného drobného kosmopolitního pěvce
nevalného nutričního významu z~čeledi Ploceidae je vyšší
než hodnota již vykrmeného měkozobého opeřence z~čeledi
Columbidae, jenž dosud setrvává na nejvyšším bodě lidského obydlí.\\[1 mm]
{\sl Lepší vrabec v~hrsti, nežli holub na střeše.}\\

\noindent
Informace neodpovídající skutečnosti se nevyznačuje dlouhými
dolními končetinami.\\[1 mm]
{\sl Lež má krátké nohy.}\\

\noindent
Informační entita charakterizovaná objektivně determinovaným
atributem negativní validity inkluduje mobilizační periferie
monodimenzionálně kontrahované.\\[1 mm]
{\sl Lež má krátké nohy.}\\

\noindent
Personifikujeme-li informaci neodpovídající skutečnosti, jsou její
dolní končetiny silně atrofovány.\\[1 mm]
{\sl Lež má krátké nohy.}\\

\noindent
Částečná immobilita muskulatury, prýštící z pasivity příslušného
cerebroorálního centra vede k souhrnu nepříznivých okolností
pro jedince přímo fatálních a zbavených veškerých chloupků a
jiného porostu.\\[1 mm]
{\sl Líná huba --- holé neštěstí.}\\

\noindent
Vodomilní obratlovci nepatrných rozměrů nejsou ničím jiným
než vodomilnými obratlovci.\\[1 mm]
{\sl Malé ryby --- taky ryby.}\\

\noindent
Argento--aurální transmutaci lze uskutečnit terminací vokálního
výstupu.\\[1 mm]
{\sl Mluviti stříbro, mlčeti zlato.}\\

\noindent
Důsledky a výtěžnost artikulační aktivity bývají zpravidla
v kompetici ohodnoceny druhou pozicí vyjádřenou kujným nerostem měsíční
barvy, zatímco zdržení se komunikace vlastního úsudku bývá upřednostňováno
a zaujímá v téže kompetici pozici první taktéž představenou kovem,
avšak barvy sluneční.\\[1 mm]
{\sl Mluviti stříbro, mlčeti zlato.}\\

\noindent
Používání akustických signálů za využití speciálního hlasivkového
krčního vaziva zaručuje výnos kujného vodivého nerostu, zatímco jejich
neužíváním dochází k zisku kovu cennějšího.\\[1 mm]
{\sl Mluviti stříbro, mlčeti zlato.}\\

\noindent
Vysoká frekvence využití hlasivkového krčního vaziva zaručuje
výnos jistého kujného nerostu, naproti tomu neužíváním zmíněného
aparátu dojde k~zisku cennějšího prvku.\\[1 mm]
{\sl Mluviti stříbro, mlčeti zlato.}\\

\noindent
Abundance šelem z čeledi Canidae implikuje exitus savce z
čeledi Leporidae.\\[1 mm]
{\sl Mnoho psů --- zajícova smrt.}\\

\noindent
Odolnost rostliny, opatřené žahavými chloupky, vůči poklesu
teploty pod 273,14 Kelvina je dokonalá.\\[1 mm]
{\sl Mráz kopřivu nespálí.}\\

\noindent
Informační kapacita jedince po sistaci srdeční činnosti je
rovna plynovým zplodinám jeho metabolismu.\\[1 mm]
{\sl Mrtvý prd ví.}\\

\noindent
Zpracovávání jakýchkoliv informací týkajících se prapředka
domácího psa občas evokuje jeho výskyt na pomezí přilehlých
polností.\\[1 mm]
{\sl My o vlku a vlk za humny.}\\

\noindent
Dne 2. února vydává drobný opeřenec své typické akustické
signály bez ohledu na s tím spojené nebezpečí letálního poklesu
tělesné teploty.\\[1 mm]
{\sl Na Hromnice musí skřivánek vrznout, i kdyby měl zmrznout.}\\

\noindent
Každého třicátého třetího dne v roce se temporální interval
mezi koncem nautického soumraku a počátkem následujícího zvětšuje
o jednu čtyřiadvacetinu střední hodnoty periody rotace zemské.\\[1 mm]
{\sl Na Hromnice o hodinu více.}\\

\noindent
Přítomnost sacharidového koncentrátu naturálního původu na
koncovém modulátoru hlasového traktu indikuje zahlcení myokardu
toxickou antivitální substancí.\\[1 mm]
{\sl Na jazyku med, v srdci jed.}\\

\noindent
Pracovní morálku zemědělce, který dne 12. března neobrací
soustavně, plánovitě a cílevědomě ornici a nedodržuje pranosticky
stanovenou agrotechnickou lhůtu, pokládáme za tak hrubé nevyhovující
a poškozující zájem společnosti, ze onoho zemědělce srovnáváme
s masožravými savci.\\[1 mm]
{\sl Na sv. Řehoře šelma sedlák, který neoře.}\\

\noindent
Kontakt malých množství sloučeniny kyslíku a vodíku se zemským
povrchem probíhající v delších časových intervalech je schopen
plně nahradit tentýž děj probíhající v intervalech kratších.\\[1 mm]
{\sl Nemusí pršet, stačí, když kape.}\\

\noindent
Pociťuje se absolutní absence výroku, jehož pravdivostní
hodnota je rovna nule.\\[1 mm]
{\sl Není šprochu, aby nebylo pravdy trochu.}\\

\noindent
Není pravda, že každé těleso, jemuž je subjektivně přisuzováno
vysoké albedo, se skládá pouze z atomů o protonovém čísle 79.\\[1 mm]
{\sl Není všechno zlato, co se třpytí.}\\

\noindent
K výskytu pojistných událostí charakterizovaných určitou
mírou materiálních, případně i jiných škod nedochází na územích
s výškovými diferencemi přesahujícími stanovenou mez, zato však
jsou vázány na jedince druhu Homo sapiens.\\[1 mm]
{\sl Neštěstí nechodí po horách, ale po lidech.}\\

\noindent
Vulgární forma plasmy je při subordinační asistenci hodnocena
celkem příznivě ve srovnání se stavem její nekontrolovatelné
dominance.\\[1 mm]
{\sl Oheň je dobrý sluha, ale špatný pán.}\\

\noindent
Inhabitanti kolumbaria, preparovaní pomoci infračervených
paprsků do stavu akceptabilního pro lidský metabolismus nevládnou
automatickou mobilitou pro perorální požití.\\[1 mm]
{\sl Pečení holubi nelítají do huby.}\\

\noindent
Vydává-li jedinec druhu Canis Domesticus krátké neartikulované
vokály, lze očekávat, že v téže chvíli nebude schopen androgenního
drajvu.\\[1 mm]
{\sl Pes, který štěká, nekouše.}\\

\noindent
Na místě, nacházejícím se v bezprostřední blízkosti zařízení
sloužícího k instalaci světelného zdroje, dopadá minimální počet
paprsků ze zdroje se šířících.\\[1 mm]
{\sl Pod svícnem bývá tma.}\\

\noindent
Spektrální odráživost všech domestikovaných savců čeledi
Felis limituje s klesajícím osvitem k nule.\\[1 mm]
{\sl Potmě každá kočka černá.}\\

\noindent
Vrubozobý pták s dlouhým krkem, obvykle domestikovaný za
účelem chovu, vydá charakteristický zvukový projev pokaždé,
když je cílenou ranou zasažen.\\[1 mm]
{\sl Potrefená husa nejvíc kejhá.}\\

\noindent
Není brzy snažit se dostihnout slovo, které se používá v~podmiňovacím
způsobu sloves.\\[1 mm]
{\sl Pozdě bycha honiti.}\\

\noindent
Vzhledem k časové ztrátě je snaha o překonání dráhy, dělící
jedince od personifikovaného kondicionálu singuláru existencionálního
slovesa, bezpředmětná.\\[1 mm]
{\sl Pozdě bycha honiti.}\\

\noindent
Produkovati slanou tekutinu, která tryská z~párových orgánů
umístěných na přední straně hlavy tak, že padá do bílého roztoku
produkovaného bučícím tvorem, je pozdě v~případě, že onen bílý
roztok zaujal místo na podlaze.\\[1 mm]
{\sl Pozdě plakat nad rozlitým mlékem.}\\

\noindent
Intenzita akustické odezvy nádoby nemalého objemu, obvykle
užívané k uskladnění kapalin obsahujících ethanol, na prudké
mechanické působení je nepřímo úměrná hustotě jejího obsahu.\\[1 mm]
{\sl Prázdný sud nejvíce duní.}\\

\noindent
Neadekvátní hodnoceni vlastního významu není v důsledku kladné
diference synchronizováno s pohybem neinerciální soustavy podél
trajektorie dané siločarami gravitačního pole.\\[1 mm]
{\sl Pýcha předchází pád.}\\

\noindent
Trajektorie pohybu způsobeného odrazy nohou mláděte létajícího
obratlovce kulminuje k maximální délce současně s východem slunce.\\[1 mm]
{\sl Raní ptáče, dál doskáče.}\\

\noindent
Střední část relativní dráhy slunce mezi nadirem a zenitem vykazuje
vyšší inteligenci nežli táž část dráhy protilehlé.\\[1 mm]
{\sl Ráno moudřejší večera.}\\

\noindent
Jedna horní končetina provádí očistu druhé horní končetiny.\\[1 mm]
{\sl Ruka ruku myje.}\\

\noindent
Počínaje překročením 48. hodiny po zahájení pobytu mimo obvyklý
okruh svého životního prostředí generují jak vodomilní obratlovci,
tak jedinci druhu Homo sapiens těkavé organické látky, jež jsou
čichovým ústrojím specificky detekovány a centrální nervovou
soustavou hrubě nepříznivě hodnoceny.\\[1 mm]
{\sl Ryba i host třetí den smrdí.}\\

\noindent
Striktní dodržování zásad občanského soužití vede k maximální
délce pěšího transferu jednotlivce, který se tak chová.\\[1 mm]
{\sl S poctivostí nejdál dojdeš.}\\

\noindent
Entropie průchodu lichokopytníka se zádovými zásobníky tuku
vodícím otvorem kovového hrotu je obecně vyšší než entropie přechodu
jedince druhu Homo sapiens, vyznačujícího se statisticky významně
vyšší životní úrovni ve srovnání s průměrem množiny jedinců,
sdílejících s nim stejný časoprostor, z tohoto časoprostoru do
jiného, charakterizovaného suverenitou hypotetické bytosti, částečně
definované tzv. náboženstvím.\\[1 mm]
{\sl Spíše projde velbloud uchem jehly, nežli bohatec vejde do
království nebeského.}\\

\noindent
Senilní jedinec přítele člověka již není schopen akceptovat
dosud neznámé dovednosti.\\[1 mm]
{\sl Starého psa novým kouskům nenaučíš.}\\

\noindent
Aplikace neurčitého výrazu tvaru $a*0$, kde $a$ může být již
trojciferné číslo, implikuje exitus domestikovaného lichokopytníka
často považovaného za symbol inteligence limitující k nule.\\[1 mm]
{\sl Stokrát nic umořilo osla.}\\

\noindent
Pohybuje-li se hladina rtuti v měrné kapiláře směrem ke středu
Země v měsíci našeho osvobození sovětskou armádou, lze očekávat,
že náš bývalý restaurační podnik přesídlí do skladiště obilovin.\\[1 mm]
{\sl Studený máj v stodole ráj.}\\

\noindent
Pohybuje-li se hladina rtuťového sloupce dolů v období 52 až 22
dní před letním slunovratem, lze očekávat výskyt biblického zapovězeného
území v silážní stavbě.\\[1 mm]
{\sl Studený máj, v stodole ráj.}\\

\noindent
Chlorid sodný získává za určité objektivní reality ve stupnici
hodnot uznávaných převážnou většinou civilizovaných jedinců vyšší
hodnocení nežli v~přírodě se mnohem vzácněji vyskytující
kujný nerost získávaný za vynaložení značného úsilí a nákladů.\\[1 mm]
{\sl Sůl nad zlato.}\\

\noindent
Přenos informace od jedince s kalorickým deficitem k jedinci
s kalorickou potřebou již saturovanou je blokován.\\[1 mm]
{\sl Sytý hladovému nevěří.}\\

\noindent
Periodická činnost s~dutou hliněnou nádobou spočívající
v~přemisťování čiré bezbarvé tekutiny bez chuti a bez
zápachu je vykonávána tak dlouho, dokud zmíněná nádoba má část
uzpůsobenou k~držení.\\[1 mm]
{\sl Tak dlouho se chodí se džbánem pro vodu, až se ucho utrhne.}\\

\noindent
Při nadměrném pohybu v delším časovém rozpětí s nádobou s držadlem
za účelem jejího naplnění bezbarvou tekutinou se velmi zvyšuje
pravděpodobnost rozdělení této nádoby na dva individuální segmenty.\\[1 mm]
{\sl Tak dlouho se chodí se džbánem pro vodu, až se ucho utrhne.}\\

\noindent
Při nadměrném zvyšování pohybu dolních končetin ve značném
časovém rozpětí za účelem dosažení naplnění sloučeninou vodíku
a kyslíku křivule s držadlem dojde jednoho dne k uvolnění molekul
spojujících tuto křivuli s oním držadlem, čímž se jmenovaný předmět
rozdělí na dva segmenty.\\[1 mm]
{\sl Tak dlouho se chodí se džbánem pro vodu, až se ucho utrhne.}\\

\noindent
Chemická sloučenina vodíku s kyslíkem produkující zvuk, jehož
intenzita se nachází pod subjektivním prahem slyšení působí erozi
na vrstvy hornin, uložených podél její trajektorie.\\[1 mm]
{\sl Tichá voda břehy mele.}\\

\noindent
Kapalná substance částečně prchavého charakteru vzniklá postupným
odšťavováním a kvašením bobulovin obsahuje souhrn výroků objektivního
charakteru vystihujících skutečnost v její reálné podstatě.\\[1 mm]
{\sl Ve víně najdeš pravdu.}\\

\noindent
Soustavnou výchovu mladého občana k obraně vlasti nelze srovnávat
s občankou, zakládající výdělečný vztah na své laktační schopnosti.\\[1 mm]
{\sl Vojna není kojná.}\\

\noindent
Místa v prostoru odpovídající polohám samic ptáků rodu Corvus
nejsou rozložena nerovnoměrně.\\[1 mm]
{\sl Vrána k vráně sedá.}\\

\noindent
Ať je trajektorie vedoucí po povrchu zemském jakákoliv, vždy
konverguje k místu o zeměpisných souřadnicích $\varphi = 12$,
$\lambda = 42$.\\[1 mm]
{\sl Všechny cesty vedou do Říma.}\\

\noindent
Gravitačně udržovaná seskupení telurických plynů jeví se
spirituální stránkou nejvyššího bodu přičitatelnosti v souboru
primitivně mytologických představ.\\[1 mm]
{\sl Vzduch --- boží duch.}\\

\noindent
Navyklá, stereotypně a pravidelně vykonávaná činnost odpovídá
horní části oděvu vyrobeného z~magneticky měkkého kovu.\\[1 mm]
{\sl Zvyk je železná košile.}\\

\noindent
Není pravdou, že existuje jedinec, který vykonal vertikální pohyb
rovnoměrně zrychlený z virtuálního bodu vytvořeného pouze pro religiózní
účely římskokatolickým klérem, pokud předtím přijímal jisté poznatky
soustavným studiem.\\[1 mm]
{\sl Žádný učený z nebe nespadl.}\\

\end{multicols}
\clearpage

% End of file
