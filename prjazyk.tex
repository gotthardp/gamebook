% Written by Petr Gotthard
% Codepage ISO-8859-2

\section{Cvičení řeči}
\begin{multicols}{\value{columnsgames}}

\subsection{Jazykolamy}

\begin{itemize}
\itemsep -3pt

\item[-] Jen mi, kmotře Petře, toho vepře nepřepepřete. Jestli mi 
toho vepře, kmotře Petře, přepepříte, pak si toho vepře, kmotře 
Petře, snězte sám.

\item[-] Letělo tři tisíce tři sta třicet tři stříbrných křepelek přes 
tři tisíce tři sta třicet tři stříbrných střech.

\item[-] Naše lomenice je ze všech lomenic ta nejlomenicovatější.

\item[-] Drbu vrbu.

\item[-] Šel pštros s pštrosicí a s pštrosáčátky Pštrosí ulicí.

\item[-] Naleju-li oleje, nenaleju-li oleje.

\item[-] Já rád játra, ty rád játra.

\item[-] Řekl řek, kolik je v Řecku řek.

\end{itemize}

\subsection{Výslovnost}

\begin{description}
\itemsep 0pt

\item[S] V~sadě se pase husa s~housaty. Sova houká v~lese. 
V~létě spíme ve stanu. Sype, les, maso, sám, snaha, sníh,
písně, miska, píská, kokos, kompas, nos.

\item[Z] Koza leze do zelí. Zuzana veze vozík. Zdeněk má
zelený balón. Je zima. Zima, zebe, zase, zebra, zívá, zisk,
vozí, kazí, kozí, zbojník, zde, zdola, zob.

\item[C] Chlapec běžel na kopec. Koníci cválají do kopce. Alice cinká 
na poklice. Cena, cihla, necky, nic, domácí, ocel, ovoce, cítí, civí,
cibule, lovec, otec, kopec.

\item[S -- C] Malé selátko cucá. Cilka má dlouhé vlasy. Slunce svítí. 
Měsíc svítí v~noci. Cosi, cesta, silnice, slepice, myslivec, ocas,
sice, svícen, slunce, tisíc, soudce.

\item[Z -- S] Na podlaze jsou zase saze. Pes veze vozík. V~zimě 
jezdíme na saních. Zase, saze, zápis, zastaví, zasadí, zásuvka, zásoba, sazenice, svízel, smazal.

\item[C -- Z] Cilka nese záclony. Někdo zcizil peníze. Cizinec zabloudil 
na ulici. Co to cinká? Cizina, cizinec, cizí, zajíc, vzácnost, záclona,
zacpal, horolezec, jezdec, vzácný.

\item[Š] Náš Míša našel v~lese šišku. V~koši byly myšky. Hoši mají tašky.
Koš, máš, náš, voláš, šumák, koše, naše, škola, škytá, výška, jíška,
lepší, ještě.

\item[Č] Anička má ráda čokoládu. Mám míč a bič. Kolotoč se točí. Čpavek 
čpí. Oč, pláč, upeč, čočka, kočka, počká, čokoláda, čouhá, čumák, často,
čepel.

\item[Ž] Želva pomalu leze. Božena má nové lyže. Žížala leží u kaluže.
Žížala, žáci, žehlí, žena, žold, lože, lyže, vlažný, dlažba, dlužen, něžný,
žně.

\item[Č -- Š] Ušáček, Vašíček, košíček, šáteček, češe, čeština, školáček, 
kašička, mašlička.

\item[Ž -- Č] Nožička, žehlička, lžička, mužíček, nožíček, čížek, kůžička, 
žehlička, nožíček.

\item[C -- Č] Čepice, cvičky, cvičí, Aničce, kočce, cvičitel, cvočky, 
čepice, tetičce, kočce.

\item[S -- Š] Sušenka, suší, sešit, sluší, snáší, soška, štěstí, šije 
se, slušně, šťastný, šosy.

\item[Z -- Ž] Zamaže, zaváže, železo, žízeň, zboží, zběžný, žíznivý,
žaluzie, žezlo.

\item[Š -- C] Švec, náušnice, myšce, lišce, o myšce, maceška, ve výšce, 
šicí, pšenice.

\item[S -- Ž] Sváže, sjíždí, složí, spolužák, sněží, snížek, užaslý,
možnosti, žalost, smaží.

\item[Č -- S] Číslo, čistí, slečna, sáček, měsíček, louskáček, sčítá součet, 
svačí, skučí, skoč.

\item[L] Nad lesem letí letadlo. Bylo bláto. Nebyla celá bílá. Slimák 
se schoval. Lípa, loďka, plakal, láme, lano, laň, louka, dál, vál, sál,
ukládá, klopýtá, kluk.

\item[R] Petr pracuje v~továrně. Traktor má silný motor. U hradu 
byl ukrutný drak. Kotrmelec, strašný, vrznout, fraška, brada, brzo, brýle, hromada, hrbol.

\item[Ř] U řeky roste řepa. Stařeček řeže dříví. Zavři pusu, Petříku. Zahřmělo.
Dřímá, truhlář, třpytka, natřásl, příběh, příchod, přijal, přejel, křižovatka.

\item[P] Pavel pracuje v~továrně. Petr propíchl peřinu. Pepa stojí 
pod lípou. Popel, pálí, Petr, pampeliška, pomalu, pec, pivo, popleta,
papoušek, páv.

\item[B] Bába Bulíková šla do lesa na houby. Béďa peče bábovku. Kubu 
bolí zoubek. Bude, bledý, bída, bouda, hubený, houba, obutý, obilí, obálka, 
Kuba, kabát.

\item[D] Máme doma dudy. David jede domů. Teta mi dala domino. Tady je 
dům. Odpoledne, domov, sudy, dudlík, dávno, datum, dech, doutnák, 
doupě, den.

\item[T] Teta stojí u stolu. Kotě v~botě tiše kouká. Potom chytí 
klubko nití. Teče, tulipán, také, kabát, nitky, kvítky, látky, kabátky,
Tonda, boty, tahat.

\item[M] Máma má malé dítě. My máme maso v~míse. Máma mele pro 
mne mák. Moje, med, mísa, Míša, mouka, máma, mnoho, málo, mák, pomocník.

\item[N] Naše Nána nosí noviny. Punťa chňapl po pečeni. Jeník měl sáně.
Noc, nový, nůše, neděle, nehoda, panenka, venku, nenapomene, nechce.

\item[H] Honza Hynek nosí hodinky. Helena hledá houby. Hbitý Haf pobíhá.
Hůl, hádá, hladí, nohy, hádanka, honem, hodnota, houska, hmota, hlava.

\item[CH] Chlapci chodí na chůdách. Hluchý lachtan delfínem je lechtán.
Chová, chechtá, chochol, buchta, chuť, chasa, chapadlo, pech, suchá.

\item[K] Karel Kovář seká sekyrou. Alenka kolébá panenku. Na poli kvete 
mák. Kůň, klepe, kytka, kykyryký, kočka, kotě, kope, koupí, liška,
lžička, puk.

\item[G] Gusta nosí gumové galoše. Gábina dala gól. Magda má guláš.
Gábina, guma, gazela, gepard, gatě, gauč, guláš, gesto, globus, magnetofon.

\item[F] Fouká vítr. Fanda foukal na flétnu. Filípek je filuta. Fujavice 
pofukuje. Fanouš, fazole, fíky, Fanda, fialky, faleš, flanel, flétna, vtip, 
fošna, fontána.

\item[V] Vašek Vítek vozí vozík. Véna veze Vandu. Kvákej: ,,kvá -- kvá 
-- kvá!'' Vana, vítr, Věra, voní, pivo, Iva, káva, konve, živé, umyvadlo, obývání.

\item[J] Jenda jí skvělá jablka. Jémine, Jana ještě nedojedla jedno jablko.
jaro, Jana, maják, jojo, majonéza, jiný, jabloň, hokej, dvě, jogurt, skvělý.

\item[DĚ] Děti dělají lodě. Dělám, dělo, viděl, děkuji.

\item[TĚ] Dítě se těší na tělocvik. Ještě, tělo, letěl, chtěl.

\item[NĚ] Při hodině někdo mluvil. Něco, němý, hodně, pěkně.

\item[DI] Na zdi visí hodiny. Dítě, chodí, diví se, vidí.

\item[TI] Tiše, tiše, dítě, spí. Tichá, utíká, tikot, vyletí.

\item[NI] Nikdo nic neví. Není, lenivý, nitě, jarní, voní.

\item[Ď] Seď, loď, hoď, viď, hleď, dívá, divný, dílo, divák, zloděj, 
vydělá, dědičný.

\item[Ť] Síť, leť, nať, nitě, tiše, potěší, tělesný, tělocvična, hostina, 
platíme, lať, kvítí.

\item[Ň] Tůň, laň, dlaň, Toník, koník, něco, nížina, nikdo, nikam, snídaně, 
básnička.

\end{description}

\end{multicols}
\clearpage

% End of file
