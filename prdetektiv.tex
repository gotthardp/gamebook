% Written by Petr Gotthard
% Codepage ISO-8859-2

\section{Příběhy takřka detektivní}

Dětem se vypráví určitý příběh. Jejich úkolem je pak přijít 
na to, co se vlastně stalo. Mohou klást pouze otázky, na něž 
je jednoznačná odpověď ,,ano'' nebo ,,ne''. Pakliže otázka nesouvisí 
s~tématem, je to možné hráčům sdělit.

\begin{multicols}{\value{columnsgames}}

\noindent
Dva bratři pracují v~cirkuse a bydlí v~maringotce. 
Jeden z~nich jednou přijde do svého, pečlivě uklizeného 
pokoje. Zavrávorá a spáchá sebevraždu. Proč?\\[1 mm]
{\sl Bratři jsou liliputáni a vystupují v cirkusu jako klauni. 
Cirkus se dostal do krize a jeden z~bratrů musí být propuštěn. 
Jeden bratr vymyslel lest: zkrátil u všeho nábytku nohy a pak 
po sobě pečlivě zahladil stopy. Druhý bratr přišel do maringotky, 
myslel si, že vyrostl a zabil se.}\\

\noindent
Poté, co muž ochutnal albatrosa a skočil z~okna. Jak to?\\[1 mm]
{\sl Manželská dvojice ztroskotala s~výletní lodí na pustém 
ostrově. Manželka se utopila a manžel ji spolu s ostatními zachráněnými 
pohřbil. Zatímco manžel truchlil, ostatní šli na lov. Protože 
nic neulovili, vykopali manželku, připravili z~ní jídlo, 
manželovi řekli, ze je to albatros. Po nějaké době šel manžel 
do restaurace a objednal si albatrosa. Když nepoznal chuť, uvědomil 
si, že předtím jedl manželku. Zbláznil se a vyskočil z okna.}\\

\noindent
Na louce sedí muž a jí květiny.\\[1 mm]
{\sl Je to v~Africe, muže zajali kanibalové. Muž věděl, 
že ho, před smrtí, čeká dlouhé mučení a chtěl si zkrátit utrpení. 
Uměl řeč domorodců a proto jim namluvil, že když sní určité množstvo 
jistých květin a potom vysloví zaklínadlo, stane se nesmrtelným. 
Domorodci si to chtěli na něm vyzkoušet a zabili ho bez mučení 
hned, jak dojedl.}\\

\noindent
Muž a žena. Najednou muž kýchnul a žena na místě zemřela.
Proč?\\[1 mm]
{\sl V cirkusové manéži probíhá vystoupení artistů na visuté hrazdě. 
Artista (muž), drží v zubech držák, na kterém balancuje artistka 
(žena). Oba jsou vysoko nad zemí, nechráněni sítí. Pojednou muž 
kýchne, držák mu vypadne, žena se zřítí na podlahu šapitó a zabije 
se.}\\

\noindent
Muž řekl ženě, že ji miluje a ona zemřela.\\[1 mm]
{\sl Byli to manželé artisti. Muž své vyznání řekl právě ve chvíli, 
když oba viseli na visuté hrazdě nechráněni sítí a muž držel 
ženu v zubech.}\\

\noindent
Kuchař připálil smaženici a zemřel.\\[1 mm]
{\sl Byl to kuchař v~ponorce. Otevřel okno, aby vyvětral.}\\

\noindent
Na louce leží mrtvý černoch.\\[1 mm]
{\sl Není to v Africe, ale např. v Malé Fatře. Dotyčný černoch 
cestoval soukromým letadlem a podle svého zvyku si chtěl před 
spaním položit pantofle před dveře.}\\

\noindent
Muhamed jde po poušti. Přijde k oáze, nahne se nad vodu 
a chce se napít. Majitel to vidí a vystřelí. Muhamed se nenapije, 
poděkuje majiteli a odejde.\\[1 mm]
{\sl Muhamed měl škytavku. O majiteli oázy nevěděl (byl ukryt 
za keříčky nebo palmou), proto ho výstřel vystrašil a škytavka 
ho od leknutí přešla.}\\

\noindent
Žena odjede do lázní, její manžel zůstane doma. Jakmile se 
po týdnu se vrátí zpět domů, manžel se zblázní. Proč?\\[1 mm]
{\sl Žena nechala manželovi doma svého oblíbeného kocourka. 
Manžel ale kocourka nesnáší a vyhodí jej z domu. Pak se manžel 
rozhodne, že koupí ženě jiného kocourka, který ji nebude poslouchat 
a to manželku zničí. V novinách si přečte inzerát, dojde do útulku 
a koupí (aniž by to věděl) toho samého kocourka. Žena se vrátí, 
kocourek se k ní žene a manžel se z toho zblázní.}\\

\noindent
Muž poslal ruku a zemřel.\\[1 mm]
{\sl Dotyčný muž před několika roky ztroskotal s jistým lodníkem 
na pustém ostrově. Před smrtí hladem je zachránilo jen to, že 
si lodník odťal ruku a společně ji snědli. Oba byli zachráněni. 
Po čase začal lodník od muže chtít, aby si i on nechal amputovat 
ruku. Měl mu ji poslat jako důkaz. Muž se však vloupal do márnice, 
ukradl tam cizí ruku a tu poslal lodníkovi. Ten se ale o podvodu 
náhodou dozvěděl a najal si vraha.}\\

\noindent
Pan Novák bydlí v~desátém patře, ale výtahem jezdí 
jen do sedmého. Proč?\\[1 mm]
{\sl Pan Novák je lilipután. Na větší číslo než je sedm
nedosáhne.}\\

\noindent
Pan Smith jezdí každý den při návratu domů výtahem do 26. 
poschodí, jen v neděli jezdí jen do 24. poschodí a zbylá dvě 
poschodí vyjde po schodech. Proč?\\[1 mm]
{\sl Pan Smith je trpaslík, na vyšší číslo než 24 nedosáhne. V 
neděli má liftboy volno.}\\

\noindent
Na poli leží mrtvola a má v~ruce ulomenou zápalku. Jak se mrtvola
na pole dostala?\\[1 mm]
{\sl Parta lidí letěla balónem a došla jim zátěž, tak losovali, 
kdo bude muset vyskočit. Mrtvola je ten, kdo prohrál.}\\

\noindent
V~bytě žijí Romeo a Julie. Julie je neustále doma, 
nikam nechodí. Romeo chodí každý den do práce. Jednoho dne přijde 
Romeo domů a nalezne Julii mrtvou. Nikdo k~nim nepřišel, 
nikdo neodešel. Proč Julie zemřela?\\[1 mm]
{\sl Julie je rybička. Romeo zapomněl před odchodem do práce vypnout 
topení v~akváriu, Julie se uvařila.}\\

\noindent
Na zemi leží mrtvý Jack.\\[1 mm]
{\sl Průvan otevřel okno a shodil akvárium. Jack byl rybička.}\\

\noindent
Pokoj, všude kolem mnoho vody a na zemi leží mrtví 
Romeo a Julie.\\[1 mm]
{\sl Rozbilo se akvárium, Romeo a Julie jsou rybičky.}\\

\noindent
Telefonní budka, v~ní mrtvý muž. Rozbité sklo, 
vyvěšené sluchátko a o budku opřený rybářský prut.\\[1 mm]
{\sl Ukazoval, jak velkou rybu chytil, pořezal se a vykrvácel.}\\

\noindent
Muž jede vlakem. Když vjede vlak do tunelu, muž 
spáchá sebevraždu (zastřelí se).\\[1 mm]
{\sl Muž byl ve městě, kde se vyléčil ze slepoty. Když vjel vlak 
do tunelu, propadl depresi, že je už zase slepý.}\\

\noindent
Otec se synem jedou autem. Dopravní nehoda, otec je mrtev, 
syna odveze sanitka do nemocnice. Chirurg, co jej má operovat, 
ale řekne: ,,Já nemohu, je to můj syn.``\\[1 mm]
{\sl Chirurg je žena, jeho matka.}\\

\noindent
Hotel, půlnoc, v~hotelu nervózní muž. Muž zvedá 
sluchátko, telefonuje a pak spokojeně usíná.\\[1 mm]
{\sl Soused vedle v~pokoji strašně chrápal. Zazvonění telefonu 
jej vzbudilo, takže náš muž může klidně spát.}\\

\noindent
Manžel jede s~manželkou k~moři. Manželka tam utone. Jakmile se
manžel vrátí, policie jej zatkne pro vraždu manželky. Aby jej
komisař usvědčil, stačil mu jediný telefonát.\\[1 mm]
{\sl Telefonoval do cestovní kanceláře, muž koupil zpáteční lístek 
jen pro sebe.}\\

\noindent
V~pokoji stojí stůl. Na stole leží karty, vedle 
nich pistole a pod stolem dvě mrtvoly.\\[1 mm]
{\sl Pokoj je v~ponorce, ve které začal docházet vzduch. 
Posádka hrála karty o to, kdo se zastřelí a kdo se udusí.}\\

\noindent
Těsně před svou smrtí schoval starý výstřední milionář v~knihovně svého sídla
do jedné z~knih dolarovou bankovku. Celý svůj majetek pak v~závěti odkázal
tomu, kdo tuto bankovku nalezne jako první. Do sídla se ihned sjeli příbuzní
a začali postupně prohledávat všechny knihy. Pouze jediný z~příbuzných šel
najisto a bankovku našel téměř okamžitě. Proč?\\[1 mm]
{\sl Knihovnu léta nikdo nepoužíval, takže podle stop v~prachu bylo poznat,
se kterou knihou se nedávno manipulovalo.}\\

\end{multicols}
\clearpage

% End of file
