% Written by Petr Gotthard
% Codepage ISO-8859-2

\section{Kvízové otázky}
\begin{multicols}{\value{columnsthindata}}

\noindent
Tvarovat je\\
a) vyrábět tvaroh\\
b) tvářit se jako tvárnice\\
\textbf{c) dávat tvar nějakému předmětu}\\

\noindent
Kompost je\\
\textbf{a) hnojivo ze zbytků}\\
b) řecky ,,kompot''\\
c) kombinovaný poštovní balík\\

\noindent
Biologicky rozložitelný je\\
a) rozložená učebnice biologie\\
\textbf{b) produkt, který se lehce rozkládá v~přírodě}\\
c) rozkládací sedadlo v~kině\\

\noindent
Čistící stanice je\\
\textbf{a) místo, kde se čistí voda}\\
b) myčka aut v~servisu\\
c) stanice metra, kde dýcháme čistý vzduch\\

\noindent
Gravitace je\\
a) čistý vzduch\\
\textbf{b) zemská přitažlivost}\\
c) latinsky ,,rytina''\\

\noindent
Recyklování je\\
a) výměna pneumatik u bicyklu\\
b) oprava dráhy pro cyklisty\\
\textbf{c) přeměna použitého materiálu na novou základní surovinu}\\

\noindent
Nezfalšovatelné označuje\\
\textbf{a) něco, co nemůžeme napodobit}\\
b) věc, kterou lze těžko provést\\
c) někoho, koho nelze nikdy podplatit\\

\noindent
Svazek je\\
\textbf{a) soubor svázaných předmětů}\\
b) zkratka svazu ekonomů\\
c) svačina z~eidamu a kopru\\

\noindent
Papyrus je\\
\textbf{a) latinské slovo označující papír}\\
b) ruský dědeček\\
c) papírový pytel plný švábů rusů\\

\noindent
Braillovo písmo je\\
\textbf{a) písmo určené nevidomým}\\
b) rukopis irského mnicha\\
c) písmo, kterým je vytištěna nejstarší kniha světa\\

\noindent
Typy jsou\\
\textbf{a) písmena nebo znaky připravené k~tisku}\\
b) stany prérijních Indiánů\\
c) nepříjemní lidé\\

\noindent
Ve vesmíru máme\\
\textbf{a) rudý obličej}\\
b) vzduch\\
c) vodu\\

\noindent
Aerodynamika je\\
a) rychlý sport\\
\textbf{b) věda o odporu vzduchu}\\
c) prehistorický motýl\\

\noindent
Piktogram je\\
a) váha menší než jeden gram\\
b) latinsky ,,zloděj obrazů``\\
\textbf{c) kresba skrývající nějaký význam}\\

\noindent
Čočky jsou\\
\textbf{a) sklíčka nahrazující brýle}\\
b) malé hnědé luštěniny\\
c) čokoládové vločky\\

\noindent
Náprava je\\
a) nádrž právě naplněná\\
\textbf{b) příčná součástka, jež drží kola}\\
c) dopravní značka ukazující doprava\\

\noindent
Vrtačka je\\
a) dívka, která všechno pokazí (zvrtá)\\
\textbf{b) nástroj sloužící k~vrtání}\\
c) středověká dětská hra\\

\noindent
Vodní hodiny jsou\\
\textbf{a) příbuzné hodin přesýpacích}\\
b) čas strávený ve vodě\\
c) stroj na měření času používaný vodníky\\

\noindent
Mikroprocesor je\\
\textbf{a) elektronický mozek}\\
b) zpěvák, který nikdy nezpívá bez mikrofonu\\
c) mikrob, který procestoval svět\\

\noindent
Integrovaný obvod je\\
\textbf{a) elektronický mozek}\\
b) závodní dráha formule 1\\
c) obvod hlavy génia\\

\noindent
Co jsou to patáky ?\\
\textbf{a) Moravsky ,,lékařské prášky``}\\
b) Vysoké jezdecké boty\\
c) Valašské krojové klobouky\\

\noindent
Jak často nastává příliv?\\
a) 1 krát denně\\
\textbf{b) 2 krát denně}\\
c) 2 krát za měsíc\\

\noindent
Čemu se říká opičí hlavolam?\\
a) Tropický pavouk\\
b) Rozcvičovací cvik mimů\\
\textbf{c) Strom blahočet chilský}\\
(má tak rostlé větve, že na něj ani opice nemohou vylézt)\\

\noindent
Jak vytvářejí včely vosk?\\
\textbf{a) Zadečkem}\\
b) Vyvrhují jej ústním otvorem\\
c) Míšením slin s~pryskyřicí stromů\\

\noindent
Šachy pocházejí z \\
a) Arábie\\
\textbf{b) Indie}\\
c) Persie\\

\noindent
První poštovní známky byly roku 1840 vydány v \\
a) Rakousku\\
b) Francii\\
\textbf{c) Anglii}\\

\noindent
Bůh Radegast symbolizoval podle starých Slovanů \\
\textbf{a) Slunce, oheň}\\
b) Bojovníka\\
c) Pokušitele\\

\noindent
Pověst vypráví, že jméno hotelu Zlatá Husa na Václavském náměstí 
vzniklo podle \\
a) Zlatého pokladu ve tvaru husy nalezeného ve sklepení\\
\textbf{b) Přihlouplé dcery tamního hostinského}\\
c) Gurmánské speciality\\

\noindent
Jezuitský řád byl založen \\
a) Sv. Augustinem\\
b) Sv. Dominikem\\
\textbf{c) Sv. Ignácem z~Loyoly}\\

\noindent
Kolik mostů bylo v~Praze na počátku 19. století\\
\textbf{a) 1}\\
b) 2\\
c) 3\\

\noindent
Šampaňské se poprvé podávalo ve století \\
a) 10.\\
\textbf{b) 17.}\\
c) 18.\\

\noindent
Muslimský svátek Ramadán trvá\\
a) Týden\\
b) 14 dnů\\
\textbf{c) Měsíc}\\

\noindent
Benátský cestovatel Marco Polo podnikl se svým otcem a~strýcem 
cestu do Číny ve století\\
a) 12.\\
\textbf{b) 13.}\\
c) 14.\\

\noindent
První film Charlieho Chaplina nese název\\
\textbf{a) Chaplin si vydělá na živobytí}\\
b) Kid\\
c) Chaplin tulákem\\

\noindent
Hora Říp je z \\
a) Žuly\\
b) Břidlice\\
\textbf{c) Čediče}\\

\noindent
Románská rotunda na vrcholu hory Říp se jmenuje rotunda\\
\textbf{a) Sv. Jiří}\\
b) Sv. Vojtěcha\\
c) Sv. Václava\\

\noindent
Hvězdu Aldebaran by jste hledali v~souhvězdí\\
\textbf{a) Býka}\\
b) Berana\\
c) Orla\\
(je nejjasnější hvězdou)\\

\noindent
Který z~uvedených kopců je nejvyšší?\\
a) Milešovka (České středohoří)\\
\textbf{b) Praha (Brdy)}\\
c) Javořice (Českomoravská vysočina)\\
(Praha 863, Milešovka 837, Javořice 837)\\

\noindent
Kokoška pastuší tobolka kvete\\
a) Modře\\
b) Žlutě\\
\textbf{c) Bíle}\\

\noindent
David Livingstone, slavný cestovatel se proslavil svými cestami 
po\\
\textbf{a) Africe}\\
b) Austrálii\\
c) Polárních končinách\\

\noindent
Známé soustroví Galapagos leží svou větší částí\\
\textbf{a) Jižně od rovníku}\\
b) Severně od rovníku\\

\noindent
Přívlastek NEVALNÝ vznikl v~souvislosti se\\
\textbf{a) Zmačkanou látkou, která se špatně válela}\\
b) Latinským ,,valere'', mít hodnotu\\
c) Záporem slovesa volit\\

\noindent
Slovo MANŽEL vzniklo ze\\
a) Staroněmeckého ,,Mannesseele'' (mužská duše)\\
b) Francouzského ,,manchelle'', což byl mužský svatební kabát 
s~dlouhými rukávy\\
\textbf{c) Staročeského ,,malženstvo'' (mladoženství)}\\
(označovalo jak manželský stav tak i~celý pár)\\

\noindent
Krápavka je \\
a) Krápníkový útvar vzniklý prokapáváním vody uvnitř staveb, 
např. tunelů\\
b) Drobná žába žijící ve středomoří\\
\textbf{c) Skleněná trubička používaná v~laboratořích pro 
odkapávání drobných kapek (pipeta)}\\

\noindent
Slovo POTENTÁT (tj. panovník nebo významný hodnostář) má následující 
původ\\
a) Každý se z~něj potentuje\\
b) Z~latinského ,,potens'' (mocný)\\
c) Z~německého ,,Bodenteider'' (ochránce země)\\

\noindent
Slovo KLUK původně označovalo\\
a) Luk\\
\textbf{b) Šíp}\\
c) Sekeru používanou pro klučení lesů\\
(neopeřený šíp, který daleko nedoletí, později přeneseno na nezralou 
mládež)\\

\noindent
Baldachýn se jmenuje podle\\
\textbf{a) Bagdádu, odkud byla dovážena na něj látka}\\
b) Čínských bálů, francouzsky ,,Bal de Chine''\\

\noindent
Cereálie jsou\\
a) Sušené celerové kostičky\\
\textbf{b) Obilniny}\\
c) Druh mořských sasanek\\

\noindent
Glycidy jsou pro organismus zdrojem\\
\textbf{a) Energie}\\
b) Vody\\
c) Glycerinu\\

\noindent
Z celkového denního energetického příjmu by měla snídaně tvořit\\
\textbf{a) 20 \%}\\
b) 30 \%\\
c) 50 \%\\

\end{multicols}
\clearpage

% End of file
